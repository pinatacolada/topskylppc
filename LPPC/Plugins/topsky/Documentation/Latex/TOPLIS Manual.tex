\documentclass[11pt,a4paper,oldfontcommands]{memoir}
\usepackage[utf8]{inputenc}
\usepackage[T1]{fontenc}
\usepackage{microtype}
\usepackage[dvips]{graphicx}
\usepackage{xcolor}
\usepackage{bookman}
\usepackage{graphicx,xcolor}
\usepackage[usestackEOL]{stackengine}
\usepackage[percent]{overpic}
\usepackage{amssymb}
\usepackage{babel,xcolor,framed,marginnote,blindtext}
\colorlet{shadecolor}{blue!10}

\newenvironment{Warn}
  {\begin{shaded}\marginnote{\fbox{Warning}}}
  {\end{shaded}}

  \newenvironment{Note}
  {\begin{shaded}\marginnote{\fbox{Note}}}
  {\end{shaded}}

\usepackage[
breaklinks=true,colorlinks=true,
%linkcolor=blue,urlcolor=blue,citecolor=blue,% PDF VIEW
linkcolor=black,urlcolor=black,citecolor=black,% PRINT
bookmarks=true,bookmarksopenlevel=2]{hyperref}

\usepackage{geometry}
% PDF VIEW
 \geometry{total={210mm,297mm},
 left=25mm,right=25mm,%
 bindingoffset=0mm, top=25mm,bottom=25mm}
% PRINT
%\geometry{total={210mm,297mm},
%left=20mm,right=20mm,
%bindingoffset=10mm, top=25mm,bottom=25mm}

\OnehalfSpacing
%\linespread{1.3}

%%% CHAPTER'S STYLE
\chapterstyle{ell}
%\chapterstyle{ger}
%\chapterstyle{madsen}
%\chapterstyle{ell}
%%% STYLE OF SECTIONS, SUBSECTIONS, AND SUBSUBSECTIONS
\setsecheadstyle{\Large\bfseries\sffamily\raggedright}
\setsubsecheadstyle{\large\bfseries\sffamily\raggedright}
\setsubsubsecheadstyle{\bfseries\sffamily\raggedright}


%%% STYLE OF PAGES NUMBERING
%\pagestyle{companion}\nouppercaseheads 
%\pagestyle{headings}
%\pagestyle{Ruled}
\pagestyle{plain}
\makepagestyle{plain}
\makeevenfoot{plain}{\thepage}{}{}
\makeoddfoot{plain}{}{}{\thepage}
\makeevenhead{plain}{}{}{}
\makeoddhead{plain}{}{}{}


\maxsecnumdepth{subsection} % chapters, sections, and subsections are numbered
\maxtocdepth{subsection} % chapters, sections, and subsections are in the Table of Contents


%%%---%%%---%%%---%%%---%%%---%%%---%%%---%%%---%%%---%%%---%%%---%%%---%%%
\begin{document}

%%%---%%%---%%%---%%%---%%%---%%%---%%%---%%%---%%%---%%%---%%%---%%%---%%%
%   TITLEPAGE
%
%   due to variety of titlepage schemes it is probably better to make titlepage manually
%
%%%---%%%---%%%---%%%---%%%---%%%---%%%---%%%---%%%---%%%---%%%---%%%---%%%
\thispagestyle{empty}

{%%%
\sffamily

\centering
\Large

~\vspace{\fill}

{\huge 
\includegraphics{img/logo.png}\\
TOPLIS
}

\vspace{2.5cm}

{\LARGE
TopSky plugin for Portugal vACC
}

\vspace{3.5cm}

User Manual
\medskip
Version 2.0
\medskip

\vspace{\fill}

October 2022

%%%
}%%%

\cleardoublepage
%%%---%%%---%%%---%%%---%%%---%%%---%%%---%%%---%%%---%%%---%%%---%%%---%%%
%%%---%%%---%%%---%%%---%%%---%%%---%%%---%%%---%%%---%%%---%%%---%%%---%%%

\tableofcontents*

\clearpage

%%%---%%%---%%%---%%%---%%%---%%%---%%%---%%%---%%%---%%%---%%%---%%%---%%%
%%%---%%%---%%%---%%%---%%%---%%%---%%%---%%%---%%%---%%%---%%%---%%%---%%%

\chapter{Introduction}

\section{Disclaimer}
Although - as its name suggests - the plugin is based on TOPLIS and the TopSky ATM system, it is in no way affiliated with or endorsed by Thales Group or NAV Portugal. Similarities between plugin features and the real system are not entirely coincidental, but the plugin can not be used as a real world training aid. ~\cite{git}

\section{Foreword}
EuroScope, a controller client developed by Gergely Csernák for the VATSIM network, was first released for public use in September 2007. One of the biggest changes in version 3.1 was the possibility for the user community to customize the program to an even higher degree than was possible before by writing their own plugins that can be used to alter the way information is presented and even create completely new functionality into the program. This allowed creating very detailed simulations of all kinds of ATC systems without making the main program overly complex. Version 3.2 expands on these possibilities, making it possible to create even better plugins.
The TopSky plugin (a.k.a. The Plugin Formerly Known As “EUROCAT 2000 E”) started out as a very small project to create a couple of customized aircraft tag items, but as more information about the real system and the possibilities with the plugin development became available, it slowly grew to include an almost complete set of tag items, tag menus, graphical elements on the radar display and some additional functionality.

\chapter{System Description}
\section{Main Window}
\includegraphics[width=15cm, keepaspectratio]{img/mainwindow.png}

Euroscope should load with some preplaced windows similar to the above configuration
\medskip

Screen resolutions other than 1920x1080 will yield different results. Larger resolutions will bring preplaced windows towards the left and middle, while smaller resolutions may potentially place windows outside the screen. It is recommended for users experiencing difficulties related to their screen size to experiment and create custom settings in the TopSkySettingsLocal file containing revised window placements adjusted for their own screen. Refer to \texttt{\detokenize{TopSky_Developer_Guide_Settings.xlsx}} for available settings
\medskip

\section{Global Menu}
\label{menu:global}
\includegraphics{img/globalmenu.png}
\medskip
The Global Menu is located on the top edge of the radar screen. It displays the current UTC time and contains a number of submenus which are explained below.

\subsection{Setup Menu}
\label{menu:set}
\includegraphics{img/Setup.png}
\medskip
Setup Menu allows for various adjustments. Each position will load its defined settings based on the active Primary Frequency.
\medskip 
Most used options are CPDLC Setting for CPDLC operations and Default Setting to reset options.
\medskip

\begin{tabular}{l l}
- Unit Settings > & Opens the \textit{\titleref{menu:unitset}}
\\- Default Setting & Resets all settings to their default values (keeps login callsign specific ones if they are active at the time). When clicked, a confirmation window will open, asking to confirm the reset.
\\- Local Settings > & Opens the \textit{\titleref{menu:localset}}
\\- Brightness Control > & Opens the \textit{\titleref{win:bcw}}
\\- Sign In… & Loads personal settings. The settings are specified in the \texttt{\detokenize{TopSkySettingsLocal.txt}} data file. When clicked, a confirmation window will open, asking to confirm the settings change.
\\- Sign Out… & Clears any personal settings and resets all settings to their default values. When clicked, a confirmation window will open, asking to confirm the settings change.
\\- CPDLC Setting… & Opens the \textit{\titleref{win:dls}}
\\- FPASD & Toggles on/off the display of flight plan tracks
\\- PDC Audible alarm & Toggles on/off playing a sound for received PDC messages
\\- CPDLC Audible alarm & Toggles on/off playing a sound for received CPDLC messages
\\- STCA Audible alarm & Toggles on/off playing a sound for STCA alerts
\\- APW Audible alarm & Toggles on/off playing a sound for APW alerts
\\- AMID & Not implemented
\\- Flight Leg & Toggles on/off the automatic display of the Flight Leg for a specified time when a track becomes assumed
\\- DAPs in Menus & Toggles on/off the display of DAPs in menus
\\- DAPs in Labels & Toggles on/off the display of DAPs in track labels
\\- RR Main > & Opens the RR Main submenu
\\- Direction Finder > & Opens the Direction Finder submenu
  \end{tabular}

\subsection*{Unit Settings submenu}
\label{menu:unitset}

This submenu can be used to change the units used in the plugin. Any changes to the settings are session-
specific only, so they will be lost when exiting EuroScope.

\begin{tabular}{l l}
    \\- Altitude & Selects the units used for altitudes and vertical rates
    - Nautical (feet, feet per minute)
    - Metric (meters, meters per second)
    \\- Flight level & Selects the units for flight levels – only applicable with metric altitudes
    - Nautical (hundreds of feet)
    - Metric (meters)
    \\- Distance & Selects the units used for distances
    - Nautical (nautical miles)
    - Metric (kilometers)
    \\- Speed & Selects the units used for speeds
    - Nautical (knots)
    - Metric (kilometers per hour)
\end{tabular}

\subsection*{Local Settings submenu}
\label{menu:localset}

This submenu allows changing some of the plugin’s settings. Any changes to the settings are session-
specific only, so they will be lost when exiting EuroScope.
\medskip

\begin{tabular}{l l}
\\- Vertical reference    & Selects the pressure reference to be used at or below the
transition altitude:
• QNH Altitude above mean sea level
• QFE Height above the aerodrome elevation
(set/check it in the adjacent box)
\\- Used equipment codes  & Selects whether to use or disregard the equipment codes
found in the flight plans:
• All Use all codes
• ICAO Use all codes when specified in ICAO format
• ICAO-alt As ICAO, but force transponder to report altitude
• None Disregard all codes

\\- ASSR codes    & Selects the transponder code source:
• Plugin Plugin data file (reverts to ESE if no codes found)
• ESE ESE file
• Range Fixed code range
\\- Groundspeed   & Selects whether to use pilot client reported ground speed or a
plugin calculated value. Normally the reported value should be
used as it is more accurate and stable, but some clients report
wrong values. If that causes problems, you can try selecting the
plugin calculated value instead
\\- Transfer confirmation & Selects when to display the Transfer Confirmation Window:
• On Always when CFL is not equal to XFL
• NotRFL When CFL is not equal to XFL unless XFL = RFL
• Off Never, any CFL value is OK
\\- CFL menu default value    & Selects the default value for the CFL menu when it is opened:
• XFL FSS or CTR: RFL if not yet reached, otherwise as below
Other: The XFL value, or current CFL value with no XFL
• CFL The current CFL value
• RFL The RFL value
\\- FPCP inhibit  & FPCP calculations start when tracks are within this time from
entering active sector
\\- STCA alert    & Selects which aircraft display the STCA alert:
• All All aircraft
• Own+Co Only assumed and coordinated aircraft
• Own Only assumed aircraft
\\- STCA alert sound  & Selects which STCA alerts trigger the alert sound:
• All All alerts
• Own+Co Only alerts with assumed and/or coordinated
aircraft involved
• Own Only alerts with assumed aircraft involved
\\- APW alert & Selects which aircraft display the APW alert:
• All All aircraft
• Own+Co Only assumed and coordinated aircraft
• Own Only assumed aircraft
\\- APW alert sound   & Selects which STCA alerts trigger the alert sound:
• All All alerts
• Own+Co Only alerts for assumed or coordinated
aircraft
• Own Only alerts for assumed aircraft
\\- METAR source  & Selects the METAR data source for the plugin windows that
display METAR data
\\- FPASD filter  & Allows filtering of displayed FPASD tracks based on sector state
• Coord Display tracks at least in the coordinated state
• Conc Display tracks at least in the concerned state
• None Display all tracks

\end{tabular}
\medskip

\subsection*{RR Main submenu}
\begin{tabular}{l l}
    - [] Rings On/Off & Toggles the range rings on/off
    \\- Point & Sets the rings centerpoint. Either click on the radar screen or
    enter the desired point in the text field. Fixes, VORs, NDBs and
    airports from the active sector file can be used as well as
    coordinates in the flight plan format (DD[N/S]DDD[E/W] or
    DDMM[N/S]DDDMM[E/W], e.g. 60N025E or 0811S00300W).
    Entering an empty text string resets the rings to be shown at
    the radar screen centerpoint.
    \\- Separation & Sets the separation between adjacent rings
    \\- Number & Sets the number of rings drawn
    \\- [] Highlight & Toggles highlight (drawn with solid line) of specified rings
    \\- Step & Sets interval of highlighted rings
\end{tabular}
\medskip

\subsection*{Direction Finder submenu}
Not operational.

\subsection{AMS menu}
\includegraphics{img/AMS.png}
\medskip
Opens the \textit{\titleref{menu:tsa}}.

\subsection{FData menu}
\includegraphics{img/FData.png}
\medskip
Opens the \textit{\titleref{menu:fpsel}} and \textit{\titleref{menu:fpwin}}.

\subsection{Tools menu}

\subsection{Tools menu}
%TODO\includegraphics{img/tools.png}
\medskip
\begin{tabular}{l l}
\\- Flight Plan Lists > & Opens the Flight Plan Lists submenu
\\- CARD… & Opens the \textit{\titleref{win:card}}
\\- SAP… & Opens the \textit{\titleref{win:sap}}
\\- Vertical Aid Window… & Opens the \textit{\titleref{win:vaw}}
\\- Message In… & Opens the \textit{\titleref{win:mi}}
\\- Message Out… & Opens the \textit{\titleref{win:mo}}
\\- CPDLC > & Opens the CPDLC submenu
\\- LAT/LONG… & Opens the \textit{\titleref{win:cur}}
\end{tabular}
\medskip

\subsection*{Flight Plan Lists submenu}
\begin{tabular}{l l}
    - [] List options bar           & Toggles the display of list options on the Global Menu
    \\- Sector List…                & Opens the Sector List
    \\- [] Informed                 & Toggles the display of informed aircraft
    \\- [] Concerned                & Toggles the display of concerned aircraft
    \\- [] Redundant                & Toggles the display of redundant aircraft
    \\- Load Factor List…           & Opens the \textit{\titleref{list:lf}}
    \\- ETWR List…                  & Opens the \textit{\titleref{list:etwr}}
    \\- <adep>                      & ETWR List departure airports filter
    \\- Uncont. List 1…             & Opens the \textit{\titleref{list:ul1}}
    \\- <filter>                    & Uncontrolled 1 List state filter
    \\- <units>                     & Uncontrolled 1 List units filter
    \\- Uncont. List 2…             & Opens the \textit{\titleref{list:ul2}}
    \\- <filter>                    & Uncontrolled 2 List state filter
    \\- <units>                     & Uncontrolled 2 List units filter
    \\- Lost List…                  & Opens the \textit{\titleref{list:ll}}
    \\- Resectorisation List…       & Opens the \textit{\titleref{list:rl}}
    \\- <lfunc>                     & Resectorisation List LFUNC filter
    \\- Traffic Mgmt. List 1…       & Opens the \textit{\titleref{list:tml1}}
    \\- <state>                     & Traffic Management List 1 flight plan state filter
    \\- <ades>                      & Traffic Management List 1 destination airports filter
    \\- <via>                       & Traffic Management List 1 route points filter
    \\- Traffic Mgmt. List 2…       & Opens the \textit{\titleref{list:tml2}}
    \\- <state>                     & Traffic Management List 2 flight plan state filter
    \\- <ades>                      & Traffic Management List 2 destination airports filter
    \\- <via>                       & Traffic Management List 2 route points filter
\end{tabular}
\medskip

When enabled, the list options bar displays “Info Conc Redu Filter Filter” on the right edge of the Global
Menu. The first three toggle the respective settings for the Sector List and are colored with the appropriate
color when enabled, and the last two are displayed in “VFR” color when the corresponding
Uncontrolled list is somehow filtered. Clicking on them opens the Flight Plan Lists submenu to change
the filtering options.
\medskip
Left-clicking <filter> cycles through “ALL” (no filtering), “ON-CONTACT” (only tracks on-contact with
anyone), “ON-CONTACT-PPOS” (only tracks on-contact with you) and “FREE” (only tracks in the free state).
\medskip
Left-clicking <units> opens a text entry box to enter a comma-separated list of aerodrome ICAO codes to
filter the list. When entered, the list will display a flight only if one of the entered codes is its departure or
destination, or the code is found in its scratchpad (OP-TEXT2).
\medskip

Left-clicking <lfunc>, <adep>, <ades> and <via> open text entry boxes to enter comma-separated lists for
controlled ID’s, ICAO codes and point names respectively to filter the affected lists.
\medskip
Left-clicking <state> toggles between “ALL” (no filtering), “SIMUL+TERM” (not started flight plans filtered),
“NOTST+SIMUL” (terminated flight plans filtered) and “SIMUL” (not started and terminated flight plans
filtered).
\medskip

\subsection*{CPDLC submenu}
\begin{tabular}{l l}
- Microphone Check      & Opens the \textit{\titleref{win:dlmcm}}
\\- Current Messages…   & Opens the \textit{\titleref{win:dlcmw}}
\\- History Messages…   & Opens the \textit{\titleref{win:dlhmw}}
\end{tabular}

\medskip

\subsection{MET menu}
\begin{tabular}{l l}
- Messages… & Opens the \textit{\titleref{win:wxcmw}}
\\- QNH/TL    & Opens the \textit{\titleref{win:wxqnh}}
\end{tabular}
\medskip

\subsection{[0]}
Not implemented (always shows a zero value).

\subsection{Info menu}
\begin{tabular}{l l}
- General Information…      & Opens the \textit{\titleref{win:gi}}
\\- Document Viewer…          & Opens the \textit{\titleref{win:dv}}
\\- NOTAM…                    & Opens the \textit{\titleref{list:notam}}
\\- Aerodrome…                & Opens the \textit{\titleref{menu:ad}}
\\- LFUNC Frequency Plan…     & Opens the \textit{\titleref{win:lfunc}}
\\- [] Airport labels         & Toggles airport labels selection
\\- [] Fix labels             & Toggles fix labels selection
\\- [] NDB labels             & Toggles NDB labels selection
\\- [] VOR labels             & Toggles VOR labels selection
\end{tabular}
\medskip
When holding <ALT>, text labels will be displayed for airports, fixes, NDBs
and VORs when the mouse cursor is placed over them. When one or more of the categories in the Info
menu is selected, only those categories will display the labels. The “Label” buttons open submenus to select
what information is shown on the corresponding labels. All the information is from the active sector file.\\

\subsection{MSG menu}
\begin{tabular}{l l}
- Notepad…              & Opens the \textit{\titleref{win:notepad}}
\\- Personal Queue…     & Opens the \textit{\titleref{win:pq}}
\\- ATC Messages…       & Opens the \textit{\titleref{win:atcm}}
\\- Prim Freq Messages… & Opens the \textit{\titleref{win:pfm}}
\\- NAT Track Messages… & Opens the \textit{\titleref{win:natm}}
\\- Text notes >        & Opens the Text notes submenu
\end{tabular}

\medskip

It is possible to insert text notes on the radar screen to act as reminders. They will stay fixed at the
geographical coordinates they are inserted to, the coordinates defining the center point of the note.

When creating a note, a text entry field opens to enter the note text. When the [Enter] key is pressed, the
note will be created at the current mouse cursor position.

The notes can be deleted one by one or all of them at the same time. When deleting one by one, the notes
are boxed to display their click areas. Clicking on one will delete the note. Pressing the [Esc] key or selecting
the “Delete...” menu item again will abort the operation.

\subsection*{Text notes submenu}
\begin{tabular}{l l}
\\- Create…     & Creates a new text note
\\- Delete…     & Deletes a single text note
\\- Delete all  & Deletes all text notes
\end{tabular}

\medskip

\subsection{[x]}
Shows the number of high priority messages in the personal message queue. These are critical failures in
the plugin code. Open the Personal Queue Window to view the messages. The number is limited to 99, and
is shown on “Global Menu Highlight” background when the window is not open.

\subsection{[x]}
Shows the number of low priority messages in the personal message queue. These are warnings about
invalid data in the plugin data files. Open the Personal Queue Window to view the messages or see the
Plugin Status submenu for more detailed information on the problem(s). The number is limited to 99, and is
shown on “Global Menu Highlight” background when the window is not open.

\subsection{STS menu}
\begin{tabular}{l l}
- Plugin Status >                & Opens the Plugin Status submenu
\\- Safety Nets Status…            & Opens the \textit{\titleref{win:sn}}
\\- Divergence Detection Status…   & Opens the \textit{\titleref{win:dds}}
\\- MTCD Status…                   & Opens the \textit{\titleref{win:mtcds}}
\\- CPDLC Default Status [ON/OFF]  & Toggles the CPDLC Default Status On/Off
\\- Runway In Use                  & Opens the \textit{\titleref{menu:ad}}
\\- Supervisory >                  & Opens the Supervisory submenu
\\- RWY line display…              & Opens the \textit{\titleref{menu:ad}}
\end{tabular}
\medskip

\subsection*{Plugin Status submenu}
Shows the version of the plugin as well as some information on the loaded data files. Each data file reports
its state with one of the following indicators:

\begin{tabular}{l l}
- OK        & File contains usable information and no faults
\\- NO DATA   & File not found or contains no usable information
\\- BAD DATA  & File contains invalid data (in “Warning” color)
\end{tabular}

Depending on the file, there are one to three of the following buttons available:\\
\begin{tabular}{l l}
- Reload                    & Reloads the data file
\\- View                    & Displays the data in the file on the radar display
\\- Save (Areas)            & Saves a snapshot of the current area activation data
\\- Save set (Maps \& MapsL) & Saves a list of currently active radar screen specific maps
\\- Load set (Maps \& MapsL) & Loads a saved list of active screen specific maps
\end{tabular}\\ 

Left-clicking the Save button will save the currently set manual activation periods as well as the
information if an area with automatic schedules is set to manual mode. The information is saved to the
“TopSkyAreasManualAct.txt” file in the same folder as the plugin dll. If the file already exists, the plugin will
ask for confirmation as the save operation will overwrite any existing data.
\medskip
Depending on the maps data file setup, the display state of some or all of the maps may be specific to each
radar screen. The Save set and Load set functions can be used to transfer the display state of these maps
from one radar screen to another.
\medskip
Right-clicking the Reload button for Settings \& SettingsL has a special purpose. It opens a text entry box to
type in a callsign whose settings should be loaded instead of the real login callsign. When entered, the
callsign will be displayed next to the “Reload” button, and whenever a VATSIM callsign change is detected,
an information popup is displayed to remind that the plugin settings are still forced to the manually
entered callsign. This feature can be used for example to use settings for different positions on different
EuroScope instances when providing top-down services, or to use settings for a specific position when
logged in with an observer/staff/supervisor callsign. Clearing the entered callsign reverts to using the
settings based on the actual login callsign.
\medskip

\subsection*{Supervisory submenu}
\begin{tabular}{l l}
- Operations Rate...     & Opens the \textit{\titleref{win:or}}
\\- Predicted Traffic... & Opens the \textit{\titleref{win:pt}}
\end{tabular}
\medskip 

\subsection{RRxxx/Off}
Opens the \textit{\titleref{menu:rr}}. If the rings are selected on,
“xxx” displays the distance between consecutive rings, otherwise “Off”.

\subsection{Mxxx-yyy}
Displays the status of the filters. If any filter is enabled and Quick Look is not toggled on, the color of the
text is “Global Menu Highlight”.
\medskip
Only the altitude filter status is shown. “xxx” displays the Lower filter value and “yyy” the Upper filter
value, in hundreds of feet.
\medskip

\subsection{S000-999}
Not implemented (shows static values).

\section{Track Presentation}
The presentation of tracks consists of the following elements:
\begin{itemize}
    \item{Aircraft position symbol}
    \item{History dots}
    \item{Prediction line}
    \item{Track label, joined to the position symbol with a leader line}
\end{itemize}

\subsection{Colors}
Most of the track presentation coloring depends on the flight sector state.
\\For controlled flights (any IFR flight or a VFR flight in ASSUMED state), the colors are as follows:\\
\begin{tabular}{l l l}
\textbf{State}          & \textbf{Color}    & \textbf{Condition}
\\Unconcerned           & “Unconcerned”     & Track will not enter the active sector
\\Notified              & “Concerned”       & Track will enter the active sector (> 15 min)
\\Coordinated           & “Coordination”    & Track will enter the active sector (< 15 min)
\\Assumed               & “Assumed”         & Track is assumed
\\Transfer Initiated    & “Assumed”         & Track is being transferred to the next controller
\\Redundant             & “Redundant”       & Track has been transferred to the next controller but is still inside the active sector
\end{tabular}

An unconcerned track can be highlighted based on rules (a combination of departure airport, route and
arrival airport) defined in plugin data files. In this case it is drawn with one of the three “Informed” colors.\\

Coordinated tracks that have not departed yet will be shown as notified instead.\\

For uncontrolled flights (VFR flights not in ASSUMED state), the colors are as follows:\\
\begin{tabular}{l l l}
\textbf{State}  & \textbf{Color}    & \textbf{Condition}
\\On Contact    & “Assumed”         & Track is on-contact (a plugin custom state) with you
\\Free          & “VFR”             & Track is not assumed or on-contact with anyone
\\Otherwise     & “Unconcerned”     &
\end{tabular}

\subsection{Aircraft position symbol}
The position symbol is drawn at the latest known position of the aircraft. The color of the symbol is the
flight sector color for an unselected track and “Track Highlight” for a selected one. A number of different
symbols are available. To begin with, there are basic shapes that tell what kind of track is in question:

\begin{tabular}{l l}
\includegraphics{img/rps_fpasd.png} & Flight plan track (position is not based on surveillance data but calculated by EuroScope)
\\\includegraphics{img/rps_coast.png} & Coasted track (no position updates in over 30 seconds, position no longer reliable)
\\\includegraphics{img/rps_psr.png} & Primary radar track
\\\includegraphics{img/rps_sec.png} & Secondary or combined radar track (uncontrolled)
\\\includegraphics{img/rps_psr+ssr.png} & Secondary or combined radar track (controlled)
\\ & ADS-B only track
\end{tabular}

An indication of an SPI (transponder ident) can be added to the secondary radar and ADS-B symbols. It
draws a cross over the symbol and prints the text “SPI” above and to the right of the symbol:

\begin{tabular}{l l}
\includegraphics{img/rps_spi.png} & Secondary radar track without DAPs with Special Position Indication
\end{tabular}

For other than the flight plan and coasted track symbols, a divergence alert will be drawn in case of a RAM
or CLAM alert. This is a circle drawn around the symbol (will not be drawn if SPI is active):

\begin{tabular}{l l}
\includegraphics{img/rps_divalert.png}& Secondary radar track without DAPs with divergence alert
\end{tabular}
\medskip

\subsection{History dots}
The history dots show the previous positions of the track. The number of displayed dots can be changed via
the \textit{\titleref{win:tcw}}. The color of the dots is the flight sector color for an unselected track and “Track
Highlight” for a selected one. History dots are not displayed for flight plan tracks.\\

\subsection{Prediction Line}
The prediction line draws the predicted ground track of the aircraft, based on its current track and ground
speed. It is a two-color line, starting with “Track Default” at the position symbol and then alternating with “Track Highlight” with every segment representing one minute of flying time. The length of the prediction line can be
changed for all tracks via the Track Control Window, or for a single track via the Prediction Line menu. The
example below shows a selected track with 5 history dots and a 3-minute prediction line. Prediction lines
are not displayed for flight plan tracks.

\includegraphics{img/rps_predline+history_selected.png}

\subsection{Track label}
There are four types of track labels that can be displayed: Standard, Reduced, Extended and Uncorrelated.
In addition, each label except the extended one has an unselected and a selected state, the selected state
being shown when the mouse cursor is over the label.
\medskip
Basically, the Standard label is shown for aircraft that are in or will enter the active sector and the Reduced
label for aircraft that will not enter the active sector. The Extended label can be opened from the Standard
or Reduced label and stays open as long as the cursor is within the label area. The Uncorrelated label is
shown for radar tracks that aren’t correlated with a flight plan.
\medskip
Refer to your setup specific documentation for detailed descriptions of the track labels.\\

\section{Flight Leg}
The Flight Leg displays the aircraft’s planned track in one-minute steps. Each one-minute-long part of the
path is colored according to the results of the MTCD and SAP processing. The following colors are possible:
\medskip
\begin{tabular}{l l}
“Urgency FL”        & MTCD and/or SAP conflic
\\“Warning FL”      & MTCD and/or SAP risk
\\“Potential FL”    & MTCD potential conflict
\\“Information FL”  & MTCD and/or SAP processing available, no conflicts or risks detected
\\“Flight Leg”      & No MTCD or SAP processing available for this part of the Flight Leg
\end{tabular}
\medskip
If the aircraft has an assigned heading or is not following its route, the predictions only go up to 10 minutes
and assume the aircraft continues on its present ground track. In this case the predicted track is shown as a
dashed line when the flight leg is displayed.
\medskip
The Flight Leg is displayed by clicking on various track label and list items depending on the setup and is
either automatically removed from display when the mouse cursor leaves the label area or must be
manually toggled off, depending on the function that was used to display it.
\medskip
The label that’s shown on each route point includes the following predefined fields
\begin{tabular}{l l}
\includegraphics{img/fleto.png}     & Estimated Time Over the point
\\\includegraphics{img/fltoc.png}   & Top of Climb
\\\includegraphics{img/fltod.png}   & Top of Descent
\end{tabular}

\section{Track Label Menus}
These menus are opened from track label fields or flight lists. Except for the confirmation windows, they
are closed automatically when a menu option is chosen or the mouse cursor leaves the menu area. Menu
items shown with (X) represent an item that has an activated and a deactivated state. With the item
activated, the item name is shown prefixed with the letter “X”. The mouse wheel can be used to scroll the
selection lists in the menus.

Many of the menus have a default item or value, displayed with inverse video. The menu usually opens so
that the default value is located under the mouse cursor for easy selection. Some menus contain items that
open folders within the menu. They show a filled triangle before the item name (upright if the folder is
closed, inverted if the folder is open). The “More” folder is opened automatically when the mouse cursor is
placed over it or if the default item is in the “More” folder, other folders must be left-clicked to open.\\

\subsection{Callsign menu}
\label{menu:cs}
\subsection*{Controlled Track}
\includegraphics{img/cm.png}
\begin{tabular}{l l l}
Assume                      & Assumes track
\\Refuse                    & Refuses the incoming transfer
\\Transfer                  & Initiates a transfer to the next sector
\\Trf \& Release             & Opens the \textit{\titleref{menu:trm}}
\\ROF                       & Sends a \textit{\titleref{dl:rof}}
\\(X)Freq                   & Toggles the Freq indicator
\\(X)Highlight              & Toggles the Callsign highlight
\\(X)S-Highlight            & Toggles the Callsign+AFL fields highlight
\\PRL                       & Opens the \textit{\titleref{menu:plm}}
\\(X)Hold                   & “Hold” opens the \textit{\titleref{menu:hold}}, “XHold” cancels a given holding clearance
\\$\blacktriangledown$ More & Shows additional less frequently used options
\\Manual Transfer           & Opens the \textit{\titleref{menu:mtm}}
\\(X)Inbound Est            & Toggles the “Inbound Est” manual alert
\\HOP                       & Initiates a \textit{\titleref{menu:hop}}
\\(X)Mark                   & Toggles the Mark indicator
\\(X)Couple                 & Uncorrelates/correlates the flight plan
\\FPL…                      & Opens the \textit{\titleref{menu:fpl}}
\\(X)Irregular              & Toggles the “Irregular” manual alert
\\Start/End CPDLC           & Starts/Ends CPDLC connection with the aircraft
\\VCI                       & Opens the \textit{\titleref{menu:vci}}
\\Squawk Ident              & Sends a “SQUAWK IDENT” CPDLC message to the aircraft
\\CPDLC Free Text           & Opens the \textit{\titleref{menu:dlftm}}
\\Free                      & Releases track
\\On Contact                & Sets track in On-Contact state*
\\(X)Missed App             & Toggles the “Missed App” manual alert
\end{tabular}
\medskip

Besides the manual alerts, none of the selectable toggle options in this menu will be transmitted to other
controllers, but the “Mark”, “Freq” and highlight selections will be seen in your other EuroScope instances.
A holding clearance is transmitted to the next controller when transferring the track. To correlate a flight
plan, first click on the “Correlate” item, and then click on the radar position symbol of the desired radar
track.
\medskip

*Clicking “On Contact” for a track with “Y” or “Z” flight rules will also automatically change the flight rules
in the VATSIM flight plan to VFR in order to make it uncontrolled. The displayed flight rules are not affected
\medskip

\subsection*{Uncontrolled Track}
\includegraphics{img/cmuncon.png}
\begin{tabular}{l l l}
On Contact          & Sets track in On-Contact state (“Assumed” color, can’t be filtered, but still uncontrolled)
\\Free              & Releases track
\\Assume            & Assumes track*
\\(X)Highlight      & Toggles the Callsign highlight
\\(X)S-Highlight    & Toggles the Callsign+AFL fields highlight
\\(X)Couple         & Uncorrelates/correlates the flight plan
\\(X)Hold           & “Hold” opens the \textit{\titleref{menu:hold}}, “XHold” cancels a given holding clearance
\\FPL...            & Opens the \textit{\titleref{menu:fpl}}
\\PRL               & Opens the \textit{\titleref{menu:plm}}
\end{tabular}
\medskip

*Clicking “Assume” for a track with “Y” or “Z” flight rules will also automatically change the flight rules in
the VATSIM flight plan to IFR in order to make it controlled. The displayed flight rules are not affected.
\medskip


\subsection*{Uncorrelated Track}
\includegraphics{img/cmuncor.png}
\begin{tabular}{l l l}
Correlate       & Correlates the radar track with the next clicked “Callsign” field
\\Create APL    & Opens the \textit{\titleref{win:apl}}
\\PRL           & Opens the \textit{\titleref{menu:plm}}
\end{tabular}
\medskip

\subsection{Transfer menu}
\label{menu:xfr}
\includegraphics{img/xfr.png}
For CPDLC connected aircraft, the menu contains options related to the transfer. Left-
clicking on the frequency button initiates the transfer (and sends the CPDLC message if
selected).

“Monitor” / “Contact” select which of the two CPDLC message types will be sent.\\
“R/T” / “CPDLC” select whether the transfer instruction is given via radio or as a CPDLC
message.

\subsection{Transfer Confirmation Window}
\label{win:xfrconfirm}
\includegraphics{img/xfrconfirm.png}
If an aircraft has a defined XFL value and hasn’t been cleared to it (CFL is not equal to XFL), attempting to
transfer the aircraft will open a Transfer Confirmation Window in the middle of the radar screen. While the
window is open it will block all other attempts to click on items elsewhere on the radar screen. Either click
on “Transfer” to transfer the aircraft regardless of the situation, or “Cancel” to cancel the transfer.\\

\subsection{Transfer \& Release menu}
\label{menu:xfrrel}
\includegraphics{img/xfrrel.png}
The Transfer \& Release menu allows specifying a release condition for a track to be
transferred. The transfer is initiated after selecting the desired condition (climb, descent,
turn or full). The release will be shown on line 0 of the track label (C for climb, D for
descent, T for turn and F for full). The transferring controller will see the label item until
the track becomes unconcerned. The receiving controller will see the item for 3 minutes
after the track is assumed.

\includegraphics{img/xfrrel.png}
For CPDLC connected aircraft, the menu contains options related to the transfer:

“Monitor” / “Contact” select which of the two CPDLC message types will be sent.

“R/T” / “CPDLC” select whether the transfer instruction is given via radio or as a CPDLC
message.

\begin{Warn}
The “Trf \& Release” option will show the release condition on the downstream side only if the
next controller is using this plugin, in other cases the transfer will be shown as a normal transfer.
\end{Warn}

\subsection{Request On Frequency message}
\label{menu:rof}

The ROF message can be used to send a request to the controller currently tracking an aircraft to transfer it
to your frequency. For the message to succeed, you must be seen as the next controller for the tracking
controller. When sent, the text “ROF” is displayed in the track label on the tracking controller’s screen.

\begin{Warn}
The “ROF” message is a feature specific to this plugin. It is an experimental feature and is not guaranteed to work all the time. When you send the message, check that it’s sent properly.
\begin{enumerate}
        \item A successfully sent message will be displayed in the \textit{\titleref{win:mow}}
        \item If there is an error or the message fails to go through, a message will be put into the \textit{\titleref{win:pq}}
\end{enumerate}
\end{Warn}

\subsection{Hold Menu}
\label{menu:hold}
\includegraphics{img/hold.png}

The Hold menu allows you to enter a holding clearance (add the aircraft to the holding
list). It displays for selection the points in the aircraft’s route that are ahead of its
current position.

Left-clicking the empty box below the waypoint list opens a text entry box to enter any
holding point name.

Left-clicking “Here” enters the present position coordinates as the holding point.

The holding point is automatically sent to your other EuroScope instances with a small
delay and can be sent to other controllers by pushing the flight strip as the information
is stored there.

\subsection{Manual Transfer Menu}
\label{menu:mxfr}
\includegraphics{img/mxfr.png}

The Manual Transfer menu allows transferring the aircraft to any controller. In the
SCHEDULED list are the controllers that are in the current sector sequence sorted in the
order the aircraft is planned to enter the controllers’ sectors, with the next controller
being the default item.

When opened, the “More” list displays all the other controllers for selection. Click on a
controller ID to start the transfer. For CPDLC connected aircraft, clicking on a controller
ID opens the \textit{\titleref{menu:xfr}}

\subsection{VCI Menu}
\label{menu:vci}
\includegraphics{img/vci.png}

Available only for CPDLC-connected aircraft and when more than one frequency has
been set up by the controller, the VCI menu allows sending a CPDLC “contact” or
“monitor” message without initiating a transfer.

The first button displays the primary frequency, left-clicking it will send the message
with that frequency.

Left-clicking the “Select Freq” button will open a text entry box to enter any other
frequency. If a valid frequency (set up as XMT TXT in EuroScope’s Voice communication
setup dialog) is entered, the message will be sent with that frequency.

“Monitor” and “Contact” are used to select the type of message to be sent.

\subsection{CPDLC Free Text Menu}
\label{menu:dlftm}

The CPDLC Free Text menu is used to send a free text CPDLC message to the aircraft. The menu contains
pre-defined messages from a data file. Left-clicking on a message sends it.

The menu closes when a message is sent or the cursor leaves the menu area.

\subsection{Prediction Line Menu}
\label{menu:prl}
\includegraphics{img/prl.png}

The Prediction Line menu allows displaying a PRL with a specific length for each aircraft
even if the PRL selection is off in the Radar Menu.

The default value is the set PRL value if available, otherwise the PRL length value from
the Track Control Window. Changing the PRL length value in the \textit{\titleref{win:trackctl}}
or changing the PRL setting in the \textit{\titleref{menu:radarm}} will delete all manually set PRL lengths.

\subsection{Sequence Number Menu}
\label{menu:seq}
\includegraphics{img/seq.png}

This menu is used to set an arrival sequence number. Values from 1 to 50 are available.

The sequence number will not be transmitted to other controllers except the next
controller (during transfer) unless the flight strip is manually sent.

\subsection{Waypoint Menu}
\label{menu:wpt}
\includegraphics{img/wpt.png}

\begin{tabular}{l l}
$\blacktriangle$  Routing & Opens the “COPN point” or “COPX point” submenu (EuroScope default item)
\\$\blacktriangle$ Arrival & Opens the “Assign STAR” submenu (EuroScope default item)
\\$\blacktriangle$ Departure & Opens the “Assign SID” submenu (EuroScope default item)
\\$\blacktriangle$ TSA Hold & Opens the TSA Hold submenu (not available if a holding clearance is active)
\\$\blacktriangle$ Hold & Opens the Hold submenu (not available if a TSA holding clearance is active)
\end{tabular}

This menu gives access to functions related to the route of the aircraft. It is used to assign
direct-to clearances, departure and arrival routes, holding clearances, and to coordinate
the sector entry/exit point.

\includegraphics{img/wptyn.png}
When an entry or exit coordination has been received, the menu opens looking like this
instead. The options are:

\begin{tabular}{l l}
$\blacktriangle$ Routing    & Opens the “COPN point” or “COPX point” submenu (EuroScope default item)
\\Accept                      & Accepts the coordination
\\Reject                      & Rejects the coordination
\end{tabular}

The submenu opened with “Routing” offers the same possibilities to accept or reject the
coordination, but also the possibility to counter-propose a different point.

\includegraphics{img/wptdl.png}

When the aircraft is CPDLC-connected and the coordination is an exit coordination, the
menu offers a choice between “R/T” and “CPDLC”. The chosen option decides how an
accepted coordination is communicated to the aircraft.

With “CPDLC” selected, when “Accept” is clicked, in addition to the coordination being
accepted, a “PROCEED DIRECT TO <point>” CPDLC message is sent to the aircraft.

\includegraphics{img/wptroute.png}

When a direct-to downlink request has been received, the menu can be used to answer it.

\begin{tabular}{l l}
Point name  & Sends a “PROCEED DIRECT TO <point>” CPDLC message
\\SBY         & Sends a “STANDBY” CPDLC message
\\UNABLE      & Sends an “UNABLE” CPDLC message
\end{tabular}   

The “R/T” / “CPDLC” selection is fixed to “CPDLC”.

\begin{Warn}
Clicking the point name will set the direct-to clearance without coordination
\end{Warn}

When there is no request in process and the aircraft has a direct-to point set, the menu
can be used to send the clearance via CPDLC. In this case the menu opens like this except
without the “SBY” and “UNABLE” buttons. Clicking the point name will send the
“PROCEED DIRECT TO <point>” CPDLC message.

\subsection*{TSA Hold Submenu}
The TSA Hold submenu allows you to enter a clearance to enter an active military area. It displays the active
and preactive TSA type areas. If a clearance already exists, the menu will only give the option to remove it
with the “XHold” item.

The clearance is automatically sent to your other EuroScope instances with a small delay and can be sent to
other controllers by pushing the flight strip as the information is stored there. A TSA hold clearance will
exclude the aircraft from all APW and SAP processing.

\subsection*{TSA Hold Submenu}
If a holding clearance already exists, the menu will only give the option to remove it with the “XHold” item.
See \textit{\titleref{menu:hold}} for other details.

\subsection{AFL Menu}
\label{menu:afl}
\includegraphics{img/afl.png}
nautical units
\includegraphics{img/aflm.png}
metric units
\includegraphics{img/kbd.png}
keyboard

This menu can be used to set the AFL value for aircraft that don’t have an altitude reporting transponder.
The default value is the previously set manual AFL value if set, otherwise the CFL value.

By default, the menu (as well as the AFL label item) is always showing nautical units, regardless of the
system units or the selected units for the aircraft. If this behavior is selected off, the list units can be
toggled with the “NAUTICAL” / “METRIC” item. There are three ways to set the AFL using this menu:

\begin{itemize}
        \item Clicking a level value in the list
        \item Clicking the text entry box below the level list and entering the value there
        \item Clicking the right-pointing triangle to open a keyboard that can be used to type in the value using the mouse. “C” clears the entry and “Ok” sets the value.
\end{itemize}

Entering a metric value will also set the aircraft’s units to metric; a nautical value will set nautical units.

The accepted manual level entry formats for the AFL, CFL and RFL menus are as follows (“n” is a number):

\begin{tabular}{l l}
“Annn” or “nnn”         & Altitude in hundreds of feet
\\“Mnnnn” or “nnnn”     & Altitude in tens of meters
\\“Mnnnnn” or “nnnnn”   & Altitude in meters
\\“Ennn”                & Height in hundreds of feet above aerodrome elevation
\\“Ennnn”               & Height in tens of meters above aerodrome elevation
\\“Ennnnn”              & Height in meters above aerodrome elevation
\end{tabular}    

Regardless of whether the entered value is in meters or feet, and altitude or height, it will be converted to altitude in feet and the result is then rounded to the nearest 100 feet.

\subsection{CFL Menu}
\label{menu:cfl}
\includegraphics{img/cfl.png}

In the track label the CFL menu is combined with the COPN altitude coordination menu
and the CFL menu opens only when the aircraft is assumed. The default value is by
default the XFL, but it can be changed to the current CFL or the RFL in the Local Settings
menu. Altitudes up to the transition altitude are prefixed with “A” in the nautical units
list and with “M” in the metric units list. QFE heights are prefixed with “E” in both lists.
Selectable values are from 500ft to FL510 with 500ft intervals up to the transition
altitude, then 1000ft intervals up to FL410 and 2000ft intervals above it.

“Visual App” / “VA” and “Clear for App” / “CA” set the corresponding approach
clearances.

The list units can be toggled with the “NAUTICAL” / “METRIC” item. There are three
ways to set the CFL using this menu:

\begin{itemize}
        \item Clicking a level value in the list or one of the two approach clearance items
        \item Clicking the text entry box between the level list and the approach clearance item and entering the value there
        \item Clicking the right-pointing triangle to open a keyboard that can be used to type in the value using the mouse. “C” clears the entry and “Ok” sets the value.
\end{itemize}

Entering a metric value will set the aircraft’s units to metric; a nautical value will set
nautical units.

The aircraft’s RFL is displayed in the place of the “NAUTICAL”/”METRIC” item with
format “R<RFL>”. Left-clicking the button still has the same effect (changes the
displayed units).

\includegraphics{img/cfldl.png}

For CPDLC connected aircraft, the menu contains “R/T” and “CPDLC” options to select
whether a level clearance is to be sent via radio or as a CPDLC message.
If a level request has been received from the aircraft, there are also “SBY” and
“UNABLE” buttons to send those messages as a reply.
- When a level request downlink has been received, the “R/T” option is deselected and
cannot be selected. The request must be replied to using CPDLC.
- When a level clearance uplink has been sent, the “CPDLC” option is deselected and
cannot be selected. If a new level clearance must be sent before there is an answer to
the uplink, it must be given via radio (doing so also closes the open uplink message).

\subsection{RFL Menu}
\label{menu:rfl}

he RFL menu allows setting the requested flight level. The operation is similar to the
AFL and CFL menus. The function for the “NEXT” button is not implemented.

\subsection{AHDG Menu}
\label{menu:ahdg}

This menu includes items to set or clear an assigned heading or a direct route and to
send a HOP. The initially highlighted heading value will be the closest one to the
assigned heading if the aircraft has one, otherwise the closest one to the aircraft ground
track (or the departure runway heading if the menu is opened from the DEP list).
Clicking on a heading value will set it as the assigned heading. The assigned heading can
also be set by typing it into the entry box, using the pop-up keyboard or by using the
AHDG vector.

“Clear” removes an assigned heading or a direct route. For CPDLC connected aircraft, it
sends the “RESUME OWN NAVIGATION” CPDLC message if the “CPDLC” option is
selected.

“Point” lets you pick a direct-to point from the radar screen. Left-click on any point to
set it as the direct-to point (available points are VORs, NDBs and waypoints, in that
priority order). Pressing the [Esc] key or clicking on any clickable data field will abort the
operation.

“HOP”, “RTI” and “TIP” are coordination functions (see below for more information). To
use them, first click on the function’s button and then select the desired value from the
list (for HOP also “Point” is available).

\includegraphics{img/ahdgdl.png}

For CPDLC connected aircraft, the menu contains additional buttons:

“R/T” and “CPDLC” select whether a heading/direct-to clearance is to be sent via radio
or as a CPDLC message.
\begin{itemize}
        \item When a heading request downlink has been received, the “R/T” option is deselected
    and cannot be selected. The request must be replied to using CPDLC.
        \item When a heading/direct-to clearance uplink has been sent, the “CPDLC” option is
deselected and cannot be selected. If a new heading/direct-to clearance must be sent
before there is an answer to the uplink, it must be given via radio (doing so also closes
the open uplink message).
\end{itemize}

“SBY” and “UNABLE” send the corresponding answers to a downlink heading request
message.

\begin{Warn}
Clicking a point on the radar screen will set the direct-to clearance without coordination 
\end{Warn}

\subsection{Handover Proposal (HOP)}
\label{win:hop}
A Handover Proposal can be used to propose non-standard transfer parameters (AHDG/Direct-to and ASP)
to the next sector. For the receiving controller a HOP is identified by coloring the callsign data field with
“Proposition” color in the label. For the sending controller the Callsign field remains “Assumed” color and
the Sector Indicator field is shown in “Proposition” color. Additionally, if there are proposed parameters
they are also colored “Proposition” in both controllers’ labels.

There are three ways to answer a HOP and all of them involve accepting all proposed parameters. If one or
more parameters are not acceptable, coordination must be done to find acceptable parameters or to revert
to standard ones. The available ways to accept the proposed parameters are:

\begin{tabular}{l l}
Callsign menu -> “Assume”               & Assumes the track
\\Callsign menu -> “ROF”                & Sends a Request On Frequency message
\\Combined Transfer menu -> “Accept”    &  Sends an Accept message
\end{tabular}

If the parameters are unacceptable to the receiving controller, the sending controller has the possibility to
modify or clear them using the appropriate menus, or to cancel the whole HOP by assuming the track.

\begin{Warn}
A HOP will only be shown correctly for controllers using this plugin. To other controllers it will be shown as a normal transfer without any special coloring of any data fields. This combined with the three possible ways to answer the HOP require the sending controller to pay special attention to the track to see what the result is.   
\end{Warn}

\begin{Warn}
If a HOP is sent to a manually selected controller, the next controller selection will be reset to the automatically calculated controller when an “ROF” or “Accept” answer is received. The correct controller must then be manually selected again for the transfer.   
\end{Warn}

\subsection{Request Tactical Instructions (RTI) / Tactical Instructions Proposal (TIP)}
\label{win:rti}
\label{win:tip}
Certain tactical data (AHDG, ASP and ARC) can be coordinated using the RTI and TIP functions. Their only difference is that RTI is used for requesting the parameters when the aircraft is inbound to your sector and your sector is the next in the sector sequence, and TIP for propose the parameters to the next sector when the aircraft is assumed. Contrary to the HOP function, these coordinations can be refused using the system, and they do not offer the aircraft for transfer.

When sent, the RTI/TIP is displayed on both controllers’ screens by displaying the requested parameter on line 0 of the track label in “Proposition” color.

To answer the RTI/TIP, left-click on the requested parameter shown above the track label or the corresponding message in the \textit{\titleref{win:miw}}. This will open the \textit{\titleref{menu:ttm}}.

\begin{Warn}
The “RTI” and “TIP” messages are features specific to this plugin. They are experimental features not guaranteed to work all the time. When you send these messages, check that they are sent properly.

\begin{itemize}
        \item A successfully sent message will be displayed in the \textit{\titleref{win:mow}} and the requested parameter being shown above the track label
        \item If there is an error or the message fails to go through, a message will be put into the \textit{\titleref{win:pq}}.
\end{itemize}
\end{Warn}

\subsection{AHDG Vector}
\label{menu:ahdgv}
The AHDG vector is another way of setting an assigned heading for an aircraft. To use the vector, left-click on the radar position symbol of the aircraft. This will start drawing the vector. When you’re satisfied with the heading value, left-click again to set it. Right-clicking will abort drawing the vector.

When the cursor is over a known point (VOR, NDB or waypoint), the name of that point is displayed instead of the heading value, and left-clicking will set a direct-to clearance to that point. To temporarily disable the known points functionality, keep the <ALT> key pressed while using the vector.

\subsection{ARC Menu}
\label{menu:arc}
\includegraphics{img/arc.png}

The ARC menu allows assigning a rate of climb or descent to the flight plan. Selectable
rates are 500-5000 ft/min (displayed in 100’s of ft/min), or 5-25 m/s. The menu units are
always the same as the units used for the aircraft in general.

Left-clicking on a value assigns it. An assigned rate can be cleared by selecting the
“Resume” item.

By default, the “+” option is selected, meaning that the clearance is a minimum rate of
climb or descent. Deselecting the “+” makes the clearance an exact rate, and selecting
the “-” option makes the clearance a maximum rate.

For “RTI” and “TIP” see the \textit{\titleref{menu:ahdg}}.

\begin{Warn}
The exact and maximum rate clearances are a feature specific to this plugin (the additional information is stored in the flight strip). To controllers not using the plugin, all assigned rate clearances will only show the rate value.
Assigned rate clearances given by controllers not using the plugin will bedisplayed as minimum rate clearances.      
\end{Warn}

\subsection{ASP Menu}
\label{menu:asp}
\includegraphics{img/asp.png}
\includegraphics{img/aspmach.png}
The ASP menu allows setting an assigned speed or Mach number. The default value will be the closest value to the assigned one if set, otherwise the plugin will suggest the closest value to the aircraft’s present speed based on the ground speed (zero wind will be assumed). The menu will initially open in IAS mode if the aircraft’s CFL is below the IAS/Mach altitude value defined in the Local Settings (FL275 by default), and in Mach mode if above it. The selectable values range from 100 to 400 knots and from Ma0.50 to Ma1.00.

The “+” and “-” options can be used to specify the clearance
as a minimum/maximum speed.

The “Resume” item clears an assigned value. For CPDLC connected aircraft, it sends the “RESUME NORMAL SPEED” CPDLC message if the “CPDLC” option is selected.

For “HOP”, “RTI” and “TIP” see the \textit{\titleref{menu:ahdg}}.

The “Resume” button below the list is replaced by a “HS” button. Clicking it will set a clearance for “high speed”, displayed as “HS” in the ASP label field (see track label definition in the local setup documentation for how to clear a value). In other setups a “high speed” clearance will show a value of 999 knots. For CPDLC connected aircraft, it sends the “NO SPEED RESTRICTION” CPDLC message if the “CPDLC” option is selected. The “Resume” button can be found at the bottom of the “More” list.

Entering a metric value will set the aircraft’s units to metric; a nautical value will set
nautical units.

\includegraphics{img/aspmachdl.png}
For CPDLC connected aircraft, the menu contains additional buttons:

R/T” and “CPDLC” select whether a speed clearance is to be sent via radio or as a
CPDLC message.

\begin{itemize}
    \item When a speed request downlink has been received, the “R/T” option is deselected and cannot be selected. The request must be replied to using CPDLC.
    \item When a speed clearance uplink has been sent, the “CPDLC” option is selected and cannot be deselected. If a new speed clearance must be sent before there is an answer to the uplink, it must be given via radio (doing so also closes the open uplink message).
\end{itemize}

SBY” and “UNABLE” send the corresponding answers to a downlink speed request.

\begin{Warn}
The minimum and maximum speed clearances are a feature specific to this plugin (the additional information is stored in the flight strip). To controllers not using the plugin, all assigned speed clearances will only show the speed value. Assigned speed clearances given by controllers not using the plugin will be displayed as exact speed clearances.
\end{Warn}

\subsection{ASSR Menu}
\label{menu:assr}
\includegraphics{img/assr.png}

The ASSR menu allows assigning an SSR code to the flight plan. To enter a new code, type it by left-clicking the numbers. “C” clears the entered value and “Ok” assigns the code if it’s a valid one. To get an automatically assigned code, clear the value and then left-click on “Ok” with the entry box left empty.

Depending on the configuration, the assigned code may be a mode S conspicuity code. To force a discrete code, make a new assignment – either manual or automatic. If an automatic assignment is requested for a flight with the conspicuity code currently assigned, the new assignment will be a discrete code.

\subsection{Combined Transfer Menu}
\label{menu:ctm}
\includegraphics{img/ctm.png}

The Combined Transfer menu displays the proposed transfer parameters for a HOP. It is opened by clicking on the AHDG, ASP or COPN/COPX items in the track label or flight list, or the list row displaying the HOP message in the \textit{\titleref{win:miw}}.

From top to bottom, the displayed values are the direct-to point, speed/Mach value, and the assigned heading value. If one or more of them are not proposed, the value will be replaced by the string “none” (the image above shows the menu for a HOP without any proposed parameters). Clicking on “Accept” will send a message to the upstream controller that the proposed parameters, if any, are all acceptable.

\subsection{Tactical Transfer Menu}
\label{menu:ttm}
\includegraphics{img/ttm.png}

The Tactical Transfer menu is used to accept, reject or apply tactical data (AHDG, ASP and/or ARC). It is opened by left-clicking on a proposed or accepted parameter in the track label. The menu displays all proposed (“Proposition” color) and accepted (sector state color) values.

Clicking on “Accept” will accept all proposed values and “Reject” will reject them. The menu is then closed.

Note that the menu displays both sent and received coordinations, but you can naturally only accept/reject the received ones and apply values for aircraft that are assumed.

Once a value is accepted, the respective label field (e.g. AHDG) will be colored “Information” until the value is set to the accepted one.

All tactical data coordinations (also any rejected ones) can be viewed in the \textit{\titleref{win:ttw}}, but they cannot be answered or applied there.

\subsection{Aerodrome Menu}
\label{menu:ad}
\includegraphics{img/ad.png}

The Aerodrome menu is used to select the aerodrome(s) for aerodrome related windows and functions. The list contains all aerodromes with runways defined in the active sector file. To select an aerodrome, left-click on it or type its identifier into the text entry box below the list.

Selection of more than one aerodrome is possible when the menu was opened from the \textit{\titleref{win:wxm}}. In this case the “All” button is available and clicking on it will select all the aerodromes in the list.

Clicking on “Ok” will confirm the selection(s) and close the menu.

\subsection{NPT Menu}
\label{menu:npt}
\includegraphics{img/npt.png}

The NPT menu is used to answer a direct-to downlink request using CPDLC. The menu
contains three options:

\begin{tabular}{l l}
Point name  & Sends a “PROCEED DIRECT TO <point>” CPDLC message\\
SBY         & Sends a “STANDBY” CPDLC message\\
UNABLE      & Sends an “UNABLE” CPDLC message\\   
\end{tabular}

The menu closes when an option is selected or the cursor leaves the menu area. If the aircraft cannot be cleared direct to the requested point but to another one, the request must be answered with “UNABLE” and a separate direct-to clearance must be given.

\begin{Warn}
Clicking the point name will set the direct-to clearance without coordination
\end{Warn}

\subsection{CPDLC Emergency Acknowledgement Menu}
\label{dleam}
\includegraphics{img/dlmayday.png}

When a CPDLC emergency message has been received, this menu is used to respond to it (if applicable), and then acknowledge the situation. When a reply is required, the menu button will read “ROGER”. Left-clicking on it will send the “ROGER” CPDLC message and close the menu. When opening the menu again (or when a reply was not required), the button reads “Ack”. Left-clicking on it will acknowledge the emergency. 

The menu is closed when the “ROGER”/“Ack” button is clicked or the cursor leaves the menu area.

\subsection{CPDLC Pilot Late Acknowledgement Menu}
\label{dlplam}
\includegraphics{img/dlplate.png}

When there is no answer to a CPDLC uplink clearance, this menu can be used to resolve the situation. “Abort” discards the uplink and “Manual WILCO” simulates reception of a WILCO message.

\subsection{Time Menu}
\label{time}
\includegraphics{img/etd.png}

The Time menu is used to set/change the time value for ATD, EOBT, ETD and SLOT fields.
Default values are:

\begin{tabular}{l l}
ATD     & Current time\\
EOBT    & Current time\\
ETD     & Current field value\\
SLOT    & Current field value if any (ATD if different from ETD), current time otherwise\\
\end{tabular}

The up/down arrows are used to change the value, “Ok” sets the time.

\subsection{Departure Sequence Menu}
\label{dqm}
\includegraphics{img/dsq.png}

The DSQ menu is used to select a specific departure sequence number to a flight. The list includes the possible numbers, and the current number is highlighted. Left-clicking on a number sets it, “Clear” removes the flight from the departure sequence.

\begin{Note}
The departure sequence number is only stored locally, it is not sent to other controllers or even to other EuroScope instances.
\end{Note}

\section{Windows}
The plugin includes a number of windows that are discussed in this chapter. All windows have the following common features:

\begin{itemize}
    \item Dragging the title bar using the left mouse button will move the window
    \item Dragging the box in the bottom right corner with the left mouse button will resize the window
    \item Left-clicking the top right corner will close the window
    \item Left-clicking the title bar will position the window on the top of other windows
    \item Right-clicking the title bar will position the window below other windows
\end{itemize}

While resizing the windows always starts from the bottom right corner, it is also possible to resize the window to the direction of the top and/or left edges. To do this, continue dragging the bottom right corner until the cursor goes past the top or left edge. As all windows have a defined minimum size, nothing will seem to happen once you reach the minimum size until the cursor crosses the opposite edge, but then the resize operation will continue normally.

Some windows contain scrollbars to select values or change the items that are displayed:

\begin{itemize}
    \item Dragging a scroll bar slider using the left mouse button will move the slider
    \item Left-clicking on the scrollbar background area outside the slider will move the slider by a predefined amount (in list windows, the view will be scrolled by the number of visible items)
    \item Right-clicking on the scrollbar background area outside the slider will position the slider to the clicked position
    \item Left-clicking on the arrow at the end of the slider will scroll the list by one line
    \item The mouse wheel can be used to scroll some scrollbars (most of the ones that have defined steps for scrolling, i.e. those with the arrows at the ends)
\end{itemize}

Other window-specific mouse function areas are explained below. All functions use the left mouse button unless otherwise specified. For each window, the way(s) to open it are listed below the chapter title.

\subsection{Radar Menu}
\label{menu:radarm}
<ALT> + Right-click anywhere on the radar screen background

\includegraphics{img/radarmenu.png}

\begin{tabular}{l l}
    Radar Menu          & Toggles keeping the menu permanently displayed\\
    Vector On/Off       & Toggles all prediction lines on/off\\
    QDM                 & Starts a new QDM vector\\
    SEP                 & Starts a new Minimum separation tool\\
    Quick Look          & Toggles function to bypass all filters and show all track labels\\
    Maps…               & Opens the Maps Window\\
    Track Control…      & Opens the Track Control Window\\
    View…               & Opens the View Window\\
    Range XXX           & Opens the Zoom Window (XXX = distance: center -> right edge)\\
    Altitude Filter X…  & Opens the Altitude Filtering Window, displays the filter status\\
    SSR Filter X…       & Opens the SSR Code Filtering Window, displays the filter status\\
    CJI Filter X…       & Opens the CJI Filtering Window, displays the filter status\\
    LAT/LONG…           & Opens the Cursor Lat/Long Window\\
    Find Track          & Not implemented\\
    Scale Marker        & Toggles the Scale Marker on/off\\
    Direction Finder    & Toggles the Direction Finder position circles or lines on/off\\
\end{tabular}

The Radar Menu closes when a selection is made or the mouse cursor leaves the menu area (unless the “Radar Menu” option is selected on).

For all the filters, it is only possible to filter out unconcerned tracks. Aircraft with transponder codes 7500, 7600 and 7700 and tracks with an active STCA, MSAW, APW or DUPE alert are also excluded from filtering. If a filter is active, the filter title in the Radar Menu will be shown in “Selected” color.

\subsection{QDM Vector}
\label{qdm}

To draw a new QDM vector:
\begin{itemize}
    \item Left-click on the “QDM” menu item
    \item Left-click on the desired start point (radar track or fixed position)
    \item Left-click on the desired end point (radar track or fixed position)
\end{itemize}

The vector’s data label is located at the end of the line. The available click spots for a radar track are the radar track position symbol and all its label items that have a mouse function.
The line end positions will attach to defined points more easily than for a random position (there is a small click area centered on the defined points). The defined points are the following, and are searched in this order:

\begin{itemize}
    \item Radar track position symbols
    \item VORs in the active sector file
    \item NDBs in the active sector file
    \item Fixes in the active sector file
    \item Airports in the active sector file
\end{itemize}

Right-clicking will abort drawing the vector.
To remove a QDM vector:

\begin{itemize}
    \item Right-click on either end point of the line (midpoint of the line for lines between two radar tracks)
\end{itemize}

To adjust a QDM vector:

\begin{itemize}
    \item Left-click on either end point. The selected end of the line will then attach to the mouse cursor.
    \item Left-click on the new desired end point (radar track or fixed position)
\end{itemize}

\subsection{Scale Marker}
\label{scale}

Radar Menu -> [] Scale Marker

\includegraphics{img/scale.png}

Displays a range scale in the bottom right corner of the radar screen.

\subsection{Minimum Separation Tool}
\label{mmtool}

The minimum separation tool displays the predicted minimum lateral separation between two radar tracks within the next 30 minutes, assuming both of them maintain their present ground tracks and speeds. Lines are drawn from the tracks’ present positions to the positions where the tracks are predicted to be at the time of the minimum separation.

To draw minimum separation lines between two radar tracks:
\begin{itemize}
    \item Left-click on the “SEP” menu item
    \item Left-click on the first radar track
    \item Left-click on the second radar track
\end{itemize}

Right-clicking will abort drawing the lines. The available click spots for a radar track are the radar track position symbol and all its label items that have a mouse function.

The minimum separation distance is by default displayed near the end of one of those lines. If the tracks are not converging, the lines will be drawn with an offline-defined length, and the label will display “DIV”.

7 sets of lines can be simultaneously drawn (plus one from the CARD). When at least one set is drawn, a SEP List Window is opened:

\includegraphics{img/seplist.png}

The window lists the tracks, the minimum predicted separation, the time to the minimum separation, the line color and a locked/unlocked indicator for each set of lines.
To remove the minimum separation lines:

\begin{itemize}
    \item Left-click on the colored box for that set of lines in the SEP List Window
    \item Right-click on a line’s end point
    \item Close the SEP List Window (this removes all minimum separation lines)
\end{itemize}

The lines will be automatically removed if one of the tracks is no longer available, or for unlocked lines, if the tracks start to diverge. If the tracks are diverging at the time the lines are created, they will be automatically locked.

To lock/unlock a set of lines:

\begin{itemize}
    \item Left-click on the box right of the color indicator for that set of lines to toggle the locked/unlocked status. For a locked set of lines, the box will be filled.
\end{itemize}

To display vertical separation information on the lines:

\begin{itemize}
    \item Right-click on the colored box for a set of lines to enable vertical separation display.
    \item Right-click again to hide the vertical separation labels
    \item Right-click once again to disable the vertical separation display
\end{itemize}

The letter “V” is shown inside the box when activated (“v” when labels have been hidden), and on both lines, two points are displayed, the first showing the point where the tracks’ vertical separation is calculated to become smaller than an offline-defined value, and the second the point after that where it is calculated to become greater. 

The calculation is done using the tracks’ current vertical speeds.

\begin{itemize}
    \item The vertical separation labels, when displayed, are similar to the minimum separation label but prefixed with “V”.
    \item If one or both points are beyond the minimum separation point, their calculation is extended forward up to an offline-defined time value.
    \item If a point is already passed or beyond the maximum displayed time, it will not be drawn.
    \item If the end point is beyond the maximum displayed time, a line will be drawn using “C\_Sep\_Vert” color from the CPA to the end point or the maximum displayed time whichever is earlier.
    \item If the tracks are not predicted to be separated by less than the defined value within the prediction time, “V=” is displayed left of the minimum separation label.
\end{itemize}

\subsection{View Window}
\label{win:view}
Radar Menu -> View...
\includegraphics{img/viewmenu.png}

The View Window lists the available views. To select a view, left-click on it. The radar screen will be refreshed to show the required area. The first three views, labeled “1”, “2” and “3” are views that can be defined on the fly. To define a view for one of them, set the screen area as desired and then right-click on the number. The number will then change to “Foreground” color to indicate that it has a view defined for it. An already defined view (“1”, “2” or “3”) can be redefined to show a different screen area just by defining it again with a right-click.

\subsection{Zoom Window}
\label{win:zoom}
Radar Menu -> Range XXX

\includegraphics{img/zoom.png}

The Zoom Window displays and enables to change the radar screen range.

\subsection{Maps Windows}
\label{win:mapsw}

Radar Menu -> Maps...

\includegraphics{img/mapsmenu.png}

The Maps Window closes when the mouse cursor leaves the window area. If this is not desired, there is a
hidden click spot in the top right corner of the menu (where the “close” button would be). Left-clicking in
that area will disable the automatic closure of the menu and display the close button, which is then used to
close the menu.

The Maps Window enables the display of predefined maps on the radar screen, some of which may be set
up with automatic activation rules. The maps are arranged to folders. Clicking on a folder name shows the
maps in that folder below the folder list.

The map names are displayed with the following colors (automatic options only available for maps with
that capability):

\begin{tabular}{l l l}
    Name & Foreground text                          & Not displayed\\
    Name & Foreground text, Foreground4 background  & Automatic (not displayed)\\
    Name & Foreground4 background                   & Automatic (displayed)\\
    Name & Foreground background                    & Displayed\\    
\end{tabular}

Left-clicking on a map name will change the state of a map one step

not displayed --> automatic (if applicable) --> displayed

Right-clicking in the other direction. Left or right double-clicking on any map name will change the states of all maps in that folder.

The “AD\_Hotspots” map in the “Aerodromes” folder is an automatically created map that contains aerodrome symbols that are used to open the \textit{\titleref{win:riuw}} and \textit{\titleref{win:ralw}}.

The “AIRPORTS”, “FIXES” (fixes with numbers filtered out), “FIXES ALL”, “NDBS” and “VORS” maps (and their “+ L” counterparts) in the “MISC” folder are automatically generated maps that contain the corresponding items with and without text labels from the active sector file. The color used for the symbols is “Auto Map Symbol” and for the labels “Auto Map Label”. Depending on the sector file setup, there can also be one or more maps named “FIXES GRP” (where “GRP” is a group name given in the sector file). These are actually NDBs or VORs in the sector file. To create such a group, put a fake navaid with the name “!GRP!” in the sector file, where “GRP” is the desired group name. All following navaids of that type will be put in the “FIXES GRP” map. The same group name can only be used once in a sector file. The items in these maps will look like the items in the “NDBS” or “VORS” maps, depending on which section the items are placed in the sector file.

In addition to plugin defined maps, the window can be used to toggle the visibility of some elements from the active sector file. These are “ARTCC HIGH”, “ARTCC”, “ARTCC LOW”, “AIRWAYS H”, “AIRWAYS L”, “SID”, “STAR”, “GEO”, “REGIONS” and “FREE TEXT”. The elements from the sector file will not show the black background when active as the plugin does not know their state. The names of these elements will be shown in grey letters to serve as reminders about this limitation.

The “AIRWAYS L” and “AIRWAYS H” folders present the airways organized into groups. The default group is “PERM”. An airway can be assigned to a custom named group by having the group name in parentheses in the airway name. Airways with “[1]”, “[2]” or “[3]” found in their names will be assigned to groups “CDR 1”, “CDR 2” and “CDR 3” respectively. Custom group names and the CDR numbers can be combined, for example airway segments named “A1(GRP)[1]” and “A2(GRP)[1]” in the sector file would be labeled “GRP CDR 1” in the maps list. Additionally, one map with the group’s name is created that contains all airways in the group regardless of the PERM or CDR status.

NAT tracks extracted from downloaded track messages (see \textit{\titleref{win:ntmw}}) can be found in the “NAT” folder. Eastbound tracks are colored East NAT Map, westbound ones West NAT Map.

The plugin data files may include additional data to the abovementioned sector file. If present, the data is activated and deactivated together with the sector file item when done via the Maps Window.

\subsection{Track Control Window}
\label{win:tcw}

Radar Menu -> Track Control...

The Track Control Window is used to set track and track label related options. Note that the PRL/Vector selection must be on in the Radar Menu to see the prediction lines. The Track Control Window closes when the mouse cursor leaves the window area. The selections are specific to the radar screen they are made on but whenever the plugin settings are reloaded either automatically (sign in/out, login callsign change when signed in) or manually the selections on all radar screens will revert to the default values.

\includegraphics{img/trackctl.png}

\begin{tabular}{l l}
    Vector          & Sets the length of the prediction line in minutes\\
    History         & Sets the number of history dots\\
    Text            & Changes the track label text size\\
    X Label Fields  & Opens/closes the corresponding Label Fields section\\
\end{tabular}

The Label Fields sections allow controlling the visibility of certain track label fields in the unselected label, and for DAP items, also in the selected label. The visibility can be set depending on the state of the fight plan (Coordinated, Assumed, Redundant or Notified/Unconcerned).

The “All” buttons toggle all the state buttons for that field on/off.

Regardless of the settings here, the necessary label fields will be displayed in certain cases (for example in case of a COPX coordination request, the COPX field will be displayed). The default label field selections are as displayed in the image.

\subsection{Altitude Filtering Window}
\label{win:afw}

Radar Menu -> Altitude Filter [ON/OFF]...

The Altitude Filtering Window is used to filter the displayed track labels based on the aircrafts’ altitudes. It closes when the mouse cursor leaves the window area.

\includegraphics{img/altfil.png}

To set the filtering limits, values from -1200ft to FL660 are available with 100ft steps up to 5000ft, then with 500ft steps up to FL660. Select the level and click on the filter on/off button to activate the limit. There are separate on/off buttons for the upper and lower limits, and it is possible to activate either one or both of them. Values at or below the transition altitude are considered to be altitudes (for example with a transition altitude of 5000ft, “050” means 5000ft and “055” means FL55).

\subsection{CJI Filtering Window}
\label{win:cji}

Radar Menu -> CJI Filter [ON/OFF]...

\includegraphics{img/cji.png}

The CJI Filtering Window closes when the mouse cursor leaves the window area.

The CJI Filtering Window is used to filter the displayed track labels based on controller ID’s. The window shows the currently online controllers. To filter a controller’s tracks, click on the controller ID in the list. A filtered ID will be shown in inverse video.

Clicking “All” will filter all controllers, and clicking “None” will clear all controller ID filters.

\subsection{SSR Code Filtering Window}
\label{win:ssr}

Radar Menu -> SSR Filter [ON/OFF]...

\includegraphics{img/ssrfil.png}

The SSR Code Filtering Window is used to filter the displayed track labels based on the aircrafts’ SSR codes. Four different codes or code ranges can be set. The boxes on the left side of the window are the code range start boxes. The input syntax is a valid SSR code to be filtered. If filtering a range of codes is needed, enter the last two digits of the last code in the range to the box on the right. Entering an empty string will clear the box. For example, to filter codes 1400-1427, enter “1400” into one of the four boxes on the left and then “27” into the box next to it. Set the filter on by clicking on the filter on/off button.

All changes to the window must be applied using the buttons in the bottom of the window to take effect.

\begin{tabular}{l l}
    Ok      & Applies the changes, closes the window\\
    Apply   & Applies the changes\\
    Cancel  & Cancels the changes\\
\end{tabular}

\subsection{Brightness Control Window}
\label{win:bcw}

Global Menu -> Setup -> Brightness Control...

\includegraphics{img/bright.png}

The Brightness Control Window allows setting the brightness for some screen colors. The Map background slider only controls the plugin created maps.

\subsection{CPDLC Setting Window}
\label{win:dls}

Global Menu -> Setup -> CPDLC Setting...

\includegraphics{img/dlsetup.png}

The CPDLC Setting Window is used to begin/end the connection to the CPDLC network, and change some CPDLC related settings.

The “Connect” button becomes available once you are connected to the VATSIM network as a controller and both the Login (four-character callsign used for the CPDLC connection) and the Logon Code (your personal password to the Hoppie network) have been entered. Left-clicking on it connects the plugin to the CPDLC network. Once a connection has been established, the button background color changes to Information and the button text changes to “Online”. Once connected, left-clicking on the button disconnects the plugin from the CPDLC network.

\begin{Note}
    It is possible to have the Logon Code pre-filled by creating a text file called “TopSkyCPDLChoppieCode.txt” in the same folder as the plugin dll (TopSky.dll). The file should contain only the logon code, nothing else. As the file contains your personal logon code, do not share it.
\end{Note}

If the VATSIM callsign is known when the window is opened, the CPDLC login callsign is pre-selected based on it. For CTR and FSS logins, the CPDLC login comes from a data file, and for others it is the beginning of the VATSIM login (which usually is the airport code). If necessary, the CPDLC login can be changed.

The “CPDLC” and “DCL/PDC” selections are used to define the offered services. For CTR and FSS logins both are selected by default, for others only “DCL/PDC” is selected.

Sending a clearance via CPDLC can be the default setting in some of the plugin menus. For this to happen, the aircraft must be above the “Min Level” (FL) specified here, the selection button for the menu in question must be on, and the CPDLC Default Status (in Global Menu->Status/STS) must be “On”. The “DCT” option does not do anything at the moment (DCT clearance via CPDLC is given using the “Point” option in the AHDG menu, so the “AHDG” option controls that as well). “Auto-accept logon” automatically accepts valid logon requests from tracks above “Min Level”.

When making changes to the “Min Level” or the menu selections, the “Ok” and “Cancel” buttons become active. Left-click on “Ok” to apply the changes or “Cancel” to abort.

\subsection{Raw Video Control Window}
\label{win:rvcw}

Global Menu -> Setup -> Raw Video Control...

\includegraphics{img/rawvideo.png}

The Raw Video Control Window closes when the mouse cursor leaves the window area.

The Raw Video Control Window controls the brightness (in general) and afterglow (how fast the radar returns fade) of the raw video radar data.

\subsection{Airspace Management Window}
\label{menu:tsa}

Global Menu -> AMS -> TSA...

\includegraphics{img/tsa.png}

This window is used for the activation and deactivation of the areas for the APW and SAP functionality. Each area can have a start time and/or an end time defined for its activation, or it can be activated without any time limits, making it active until deactivated manually. Additionally, lower and upper altitude limits are given. An area can have activation schedules defined in the area data file. Such areas will be automatically activated as long as their “Auto” option is selected ( “A” in the “Auto” column). The “Auto” option cannot be selected for areas that don’t have an activation schedule defined in the area data file.

Dates will be shown in the format “yymmdd” and times in “hh:mm” and they must be entered in the same format. Entering an empty string for a date will clear it and the related time value and vice versa. When entering a time or date value to an empty field, the other value is automatically set to the current time/date value. Entering an empty string to the Map Text, Lower or Upper fields will reset the value to the default one from the data file.

Altitudes are shown in hundreds of feet if at or below the transition altitude, otherwise in flight levels. They must be entered in the same format.

An area’s activation status can be inactive, pre-active or active. A pre-active area is an area that will become active within 30 minutes and is shown in yellow text on a gray background. An active area is shown with yellow text on a blue background. The APW system will not alert for a pre-active area, but for the SAP system a pre-active area is considered as being active.

The mouse click areas of the Airspace Management Window:
\begin{itemize}[\textbullet] 
        \item Sorting option text (e.g. “Start date/time”) Opens a pop-up menu to select a sorting option for the list 
        \item Right-click to open an area pop-up menu
        \item Other fields Left-click to edit field (when edit function active)
        \item “Ok” button Applies the changes, closes the window
        \item “Apply” button Applies the changes
        \item “Cancel” button Cancels the changes 
\end{itemize}

The sorting pop-up menu contains the following items:
\begin{itemize}[\textbullet] 
        \item Start Date Sorts based on the Start Date/Time, earliest first
        \item Name Sorts alphabetically based on the Name field
        \item Map Text Sorts alphabetically based on the Map Text field 
\end{itemize}
With the area pop-up menu opened, the area text row background changes to black. The menu contains the following items:
\begin{itemize}[\textbullet] 
        \item ACTIVATE Clears any activation times and activates the area
        \item DEACTIVATE Clears any activation times and deactivates the area
        \item AUTO If an activation schedule is found in the area data file, sets the
        \item area to be activated automatically
        \item VALIDATE Not implemented
        \item EDIT Allows to change the area parameters
        \item COPY Not implemented
        \item DELETE Clears any activation times, returns label and altitude limits to their default values and deactivates the area
\end{itemize}
After any selection from the pop-up menu, “Ok”, “Apply” or “Cancel” must be selected to apply or cancel the selection. 

Preactive and active areas are displayed on the radar screen. The area border is drawn using a predefined color and it may be filled as well. A predefined text label may also be displayed, showing information about the area. A very small “+” symbol will be drawn at that location. By holding the left mouse button down on that symbol, a full area label will be displayed, showing:

\begin{center}
        Name\\ 
        Map text\\
        Upper level limit\\
        Start time --------- End time\\
        Lower level limit\\time in minutes until the area becomes active
\end{center}

\subsection{Flight Plan Selection Window}
\label{win:fpsw}

Global Menu -> FlightData/FData -> Flight Plan Selection...

\includegraphics{img/fpsel.png}

The Flight Plan Selection window is used to search for flight plans based on any combination of callsign, assigned transponder code and departure airport. All flight plans that are a match with all given information will be listed in the \textit{\titleref{win:fpw}}. and can be viewed using its “Prev” and “Next” buttons.

Clicking “Ok” will do the search and open the Flight Plan Window if it was closed, “Cancel” will clear the fields as well as any previously created flight plan list for the Flight Plan Window.

\subsection{Flight Plan Window}
\label{win:fpw}

Global Menu -> FData -> Flight Plan Window… 
Flight Plan Selection Window -> Create a list of one or more flight plans 
Callsign menu -> More -> FPL… 
“Open FPL Window” tag function

\includegraphics{img/fpl.png}

The Flight Plan Window displays flight plan data as well as some system data related to the flight plan. It also allows creating new flight plans and modifying existing ones. The data in the window is not refreshed automatically; the time when the data was fetched is displayed in the top left corner.

The data fields show the following information:

\begin{tabular}{l l}
    Callsign    & Callsign of the aircraft\\
    FRUL        & Flight rules (I, V, Y or Z)\\
    FTYP        & not available\\
    NRAC        & Number of aircraft\\
    ATYP/W      & Aircraft type and wake turbulence category\\
    EQCST       & Displays aircraft equipment status for certain equipment (W, Y, U, R and P) - “EQ” equipped, “NO” not equipped, “UN” unknown.\\
    EOBD        & Estimated Off-Block Date\\
    EOBT        & Estimated Off-Block Time\\
    EQUIP       & Equipment list - For flight plans with FAA equipment codes, a rough conversion to ICAO is shown in parentheses\\
    ADEP        & Departure aerodrome\\
    ADES        & Destination aerodrome\\
    EET         & Estimated Elapsed Time\\
    ALTN        & Alternate aerodrome(s)\\
    ETD         & Estimated Time of Departure\\
    ETB         & Estimated time to enter your sector\\
    TAS         & True Air Speed\\
    RFL         & Requested Flight Level\\
    RTE         & Route\\
    Other       & Flight plan remarks field\\
    SUP         & Supplementary information (endurance, PIC name)\\
\end{tabular}

The following system information regarding the flight plan is shown unless the initial plan is displayed:

\begin{tabular}{l l}
    PSSR    & Previous SSR code\\
    ASSR    & Assigned SSR code\\
    PEL/CFL & Planned Entry Level or Cleared Flight Level, depending on the flight’s state\\
    Type    & Type of flight plan (APL or FPL)\\
    Status  & Status of the flight plan\\
    ETN     & Estimated time to COPN\\
    COPN    & Entry coordination point\\
    ETX     & Estimated time to COPX\\
    COPX    & Exit coordination point\\
    LFUNC   & Controller who is currently tracking the aircraft\\
    0/0     & Number of the displayed FPL in the list / total number of flight plans in the list\\
\end{tabular}

The following buttons are available:

\begin{tabular}{l l}
    CRE         & Create a new full flight plan (FPL)\\
                & Editable fields will be highlighted\\
                & If a flight plan is being displayed, all data fields keep their values so a new flight plan can be created using an existing one as a base. If not, default values will be set to FRUL, NRAC, EQUIP and Other fields\\
                & “Apply” creates the FPL, “Cancel” aborts the operation\\
    MOD         & dify the currently displayed FPL\\
                & Available fields will be highlighted\\
                & “Apply” modifies the FPL, “Cancel” aborts the operation\\
    Find        & Find a flight plan\\
                & Enter Callsign to find the flight plan, “Cancel” aborts the operation\\
    Probe       & Not implemented\\
    Force       & Force this aircraft to be included in the MTCD and SAP processing regardless of its sector state or any inhibition settings in the MTCD Status Window\\
    ACT         & Not implemented\\
    SAC         & Enter a slot time\\
                & Enter the time to the ETD field, “Cancel” aborts the operation\\
    CREAPL      & Create a new abbreviated flight plan (APL)\\
                & Editable fields will be highlighted\\
                & “Apply” creates the APL, “Cancel” aborts the operation\\
    MODAPL      & Modify the currently displayed APL\\
                & Editable fields will be highlighted\\
                & “Apply” modifies the APL, “Cancel” aborts the operation\\
    CONV        & Convert an APL to an FPL\\
                & Editable fields will be highlighted\\
                & Default values will be set to FRUL, NRAC, EQUIP and Other fields\\
                & “Apply” converts the APL, “Cancel” aborts the operation\\
    Unforce     & Cancel the forced inclusion of this aircraft in the MTCD and SAP processing\\
    Terminate   & Not implemented\\
    CNL         & Not implemented\\
    RTE         & Opens the Complete Route Window\\
    Prev        & Selects the previous flight plan in the list (see Flight Plan Selection Window)\\
    Next        & Selects the next flight plan in the list (see Flight Plan Selection Window)\\
    INI         & View the initial flight plan\\
                & Complete Route Window is closed if it was opened\\
                & Flight plan refresh time and system information will not be shown\\
                & “CRE” starts to create a new flight plan based on the displayed initial flight plan, “Refresh” or “Cancel” shows the current flight plan again\\
    Refresh     & Refreshes the displayed information\\
    Apply       & Apply changes that were made\\
                & Any errors will be displayed in the bottom row of the window\\
    Cancel      & Cancel any changes and quit the current operation\\
\end{tabular}

In the plugin the only difference between an APL and an FPL is that an APL can only contain the Callsign. If it contains any other information (controller-assigned values such as ASSR, CFL, etc. are not taken into account), it will be considered to be an FPL.

For the accepted CFL entry formats, see \textit{\titleref{menu:afl}}.

\subsection{Complete Route Window}
\label{win:crw}

Flight Plan Window -> “RTE” button

\includegraphics{img/route.png}

Displays the expanded route of the FPL currently shown in the \textit{\titleref{win:fpw}}. Already passed points are shown with “Overflown” color background. Points still ahead show the estimated time over the point and the calculated flight level (as calculated by EuroScope) below the point name.

\subsection{Create APL Window}
\label{win:apl}

Callsign menu (uncorrelated track only) -> “Create APL” item

\includegraphics{img/apl2.png}

The bottom area of the window will show status and error messages.

Enter all the available information (the only required field is the callsign) and click “Ok” to create the APL. “Cancel” will clear all the fields.

If the SSR field is left empty, a code will be automatically assigned from a dedicated APL code range. If no codes are available, an error message will be shown and a code must be manually entered.

For the accepted CFL entry formats, see \textit{\titleref{menu:afl}}.

Left-clicking on the FRUL field toggles the flight rules between “I” (IFR) and “V” (VFR).
When the APL is successfully created, it is automatically assumed or set as on-contact depending on the plugin setup.

\subsection{Stack Manager Window}
Holding List... -> “HPT” item

\includegraphics{img/stackman.png}

To help with controlling holding aircraft, Stack Manager Windows can be opened. They give a quick look into the vertical positions of aircraft that have been given a holding clearance to the holding fix associated with the window in question.

The window is split into three columns:

\begin{tabular}{l l}
    Approaching Hold:   & Aircraft approaching the holding area\\
                        & (more than 5 minutes flying time away from the holding fix)\\
    Hold:               & Aircraft in the holding area\\
    Leave Hold:         & Aircraft cleared to leave the holding\\   
\end{tabular}

The window displays the aircraft labels at their cleared flight levels. For each aircraft the callsign, the vertical speed arrow if not in level flight (“\#” in “Warning” color for aircraft without altitude information), the RVSM capability indicator if applicable, and the CFL is displayed. If there are more than one aircraft with the same CFL, only one callsign will be shown. A “ + ” symbol in “SMW Overlap Box” color after the CFL value indicates that there are more flights to be displayed. Clicking on the symbol will display a window with all the callsigns with that CFL.

For each aircraft an altitude box will be drawn that extends from AFL to CFL. Normally the color of the box is “SMW Level Band”, but in the Hold and Leave Hold columns if it is closer than 300ft or overlaps any other aircraft’s box, the color will be “SMW Overlap”. For an aircraft with a CLAM alert, the box will be in “SMW Overshoot” color. A flight level reserved for an overflight is shown in “SMW Overflight” color. The altitude box of aircraft in the Leave Hold column extends into the Hold column as well.

Aircraft will be automatically removed from the Leave Hold column after 10 minutes, but they can also be manually removed by right-clicking the callsign.

Any number of Stack Manager Windows can be opened to monitor multiple holdings at the same time. For CFLs at or below the transition altitude, the aircraft label is placed on the numerical value of the CFL, but the altitude box is always based on flight levels.

The mouse click areas of the Stack Manager Window:

\begin{tabular}{l l}
    Slider or view background area  & Drag to adjust the lowest shown FL\\
    Aircraft callsigns              & Open \textit{\titleref{menu:cs}} *\\
    CFL                             & Open \textit{\titleref{menu:cfl}} **\\
    + symbol                        & Open window to view all callsigns with that CFL\\
    FL numbers                      & Toggle overflight status for that level\\   
\end{tabular}

* Right-clicking a callsign in the Leave Hold column immediately removes the aircraft from display.
** When opened from a Stack Manager Window, the default value in the CFL menu is the current CFL regardless of the setting in the \textit{\titleref{menu:localset}}.

\subsection{CARD}
\label{win:card}

Global Menu -> ControlTools/Tools -> CARD..

\includegraphics{img/card.png}

The CARD window presents the MTCD conflicts and conflict risks. It also allows setting some of the related parameters. It is divided into two parts:

On the left is the list area which includes all the detected conflicts and shows the concerned aircrafts’ callsigns. A conflict or potential predicted conflict has an “Urgency” colored label and a risk or potential risk of conflict has a “Warning” one. Potential conflicts are displayed with “Potential” colored labels. The conflict numbers are shown in “CARD Conflict Number” color. An “Urgency” colored background on a callsign means that the aircraft also has a Short Term Conflict Alert. A highlighted callsign will be highlighted in this list as well. An acknowledged conflict (shown in “Unconcerned” color) will not display the MTCD warning in the track label. An acknowledged problem will be automatically de-acknowledged if the predicted minimum separation decreases by a predefined amount.

Hovering the mouse cursor over a conflict label will display the conflict on the radar screen.

On the right is the graphical display area that gives an overview to the severity and timeframe for each conflict. On the vertical (distance) axis the conflicts are placed to the predicted minimum separation and on the horizontal (time) axis the label is placed so that the left edge of the conflict number is at the time of closest point of approach. A line in “CARD Time Vector” color, extending to the left from the label, marks the time when the separation will decrease below the prediction distance. For fast closure rates the time from prediction distance to CPA may be too short to display the line. The conflict labels have the same mouse functions as the ones in the list area. If the label position would be outside the maximum time displayed in the window, the label is positioned at the maximum displayed time with a “Foreground” colored edge.
The area below 5nm distance is drawn in “Field Highlight” color.

The mouse click areas of the CARD window:

\begin{tabular}{l l}
    “Timeloss” / “Distance”     & Opens a pop-up menu to select the sorting option\\
                                & (Time to start of conflict / Predicted minimum separation)\\
    “View” menu label           & Toggle the View menu.\\
    Vertical slider             & Drag to adjust the distance scale\\
    Horizontal slider           & Drag to adjust the time scale\\
    Diamond on vertical axis    & Drag to adjust prediction distance    \\
    Diamond on horizontal axis  & Drag to adjust prediction time\\
    Triangle on horizontal axis & Drag to adjust warning time\\
    Conflict number labels      & Left-click to open Mark/ACK menu\\
                                & Right-click to toggle SEP (minimum separation lines)\\
\end{tabular}

From the “View” menu it is possible to toggle various MTCD related options:

\begin{tabular}{l l}
    List                & Toggles display of the list area\\
    Graphic             & Toggles display of the graphical area\\
    Grid                & Toggles a nm/min grid on the graphical area\\
    Risk                & Toggles display of MTCD risks of conflict\\
    Potential Predicted & Toggles display of MTCD potential predicted conflicts *\\
    Potential Risk      & Toggles display of MTCD potential risks of conflict *\\
    Potential           & Toggles display of MTCD potential conflicts *\\
    PLC                 & Toggles display of planner controller conflicts\\
                        & (conflict starts later than the triangle displayed on the time axis)\\
    MTCD Ind            & Toggles the display of the MTCD indicator on the track label\\
    Notif               & Toggles whether tracks in the notified state are considered for MTCD\\
    Unco                & Toggles whether unconcerned tracks are considered for MTCD\\
    Future              & Toggles whether flight plan tracks are considered for MTCD (flight plan tracks must be at least in the coordinated state regardless of the “Notif” and “Unco" selections)\\
\end{tabular}

* The availability of these items in the menu depends on the plugin setup

From the “Mark/ACK” menu it is possible to toggle conflict-specific options:

\begin{tabular}{l l}
    Mark All    & Toggles marking the Callsigns and AFL’s of the concerned tracks with “CARD Mark All” color\\
    Mark Own    & Toggles marking the Callsigns of the concerned tracks with “CARD Mark Own” color\\
    ACK         & Toggles acknowledgement status\\
    Address     & Not implemented\\
\end{tabular}

\subsection{SAP Window}
\label{win:sap}

Global Menu -> ControlTools/Tools -> SAP...

\includegraphics{img/sap.png}

The SAP window lists the aircraft that have SAP conflicts (can be set to also show risks). The list shows the aircraft callsign, the area it will enter and the coordinates and the time when it will enter it. Only the first entered area will be shown if the aircraft is predicted to enter more than one active area. Placing the mouse cursor over an aircraft line will show the aircraft’s route on the radar screen up to the first point of entering an active area or the first point where there is a risk of doing so if risks are selected to be displayed on the window. The list is sorted according to the entry time field, with the earliest time on top.

\subsection{Vertical Aid Window}
\label{win:vaw}

Global Menu -> ControlTools/Tools -> Vertical Aid Window...

\includegraphics{img/vaw.png}

The Vertical Aid Window shows the predicted vertical trajectory of the selected aircraft (ASEL), starting from its current position (marked with a dot in “VAW Track Position” color), and the trajectories of all aircraft conflicting with it. The trajectories are displayed as calculated by EuroScope. The background color of the path area is by default the “Inactive Sector” color, but for the time the aircraft is predicted to be inside your sector the color will be the “Active Sector” color. Other sector boundary crossings are displayed with vertical lines in “VAW Sector Limits” color. The trajectory of the ASEL aircraft is drawn in “VAW Profile” color and the conflicting aircrafts’ trajectories in “Urgency” (conflict or potential predicted conflict), “Warning” (risk or potential risk of conflict) or “Potential” (potential).

It is also possible to send PEL and XFL coordinations and set the CFL from this window using the two buttons:

\begin{tabular}{l l}
    CFL/PEL     & Opens the PEL or CFL menu depending on sector state\\
    View        & Toggles the View menu\\
    XFL         & Opens the XFL menu\\
\end{tabular}

From the “View” menu it is possible to toggle various MTCD related:

\begin{tabular}{l l}
    Risk                    & Toggles display of MTCD risks\\
    Potential Predicted     & Toggles display of MTCD potential predicted conflicts *\\
    Potential Risk          & Toggles display of MTCD potential risks of conflict *\\
    Potential               & Toggles display of potential conflicts *\\
    Notif                   & Toggles whether tracks in the notified state are shown\\
    Unco                    & Toggles whether unconcerned tracks are shown\\
\end{tabular}

* The availability of these items in the menu depends on the plugin setup

The PEL and XFL values are displayed in “Coordination” color, or in “Proposition” if being coordinated. After a refused coordination, the original value is shown in “Warning” color. The CFL value is displayed as a horizontal line across the screen.

\subsection{Message In Window}
\label{win:miw}

Global Menu -> ControlTools/Tools -> Message In...

\includegraphics{img/miw.png}

The Message In Window shows received coordination messages, sorted by time, with the newest ones at the top of the list. For the ones requiring an answer, it is possible to send it either from this window, the aircraft track label or any aircraft list where the relevant items are displayed. The messages will be automatically removed from the list when the track becomes Unconcerned, or for some messages, also based on a specific event.

The available message types are:

\begin{itemize} 
    \item “From <SI> proposes: <Callsign> is coming to your frequency”
         \begin{itemize} 
            \item Displayed when the track is being transferred to you
            \item Removed when the transfer is complete or cancelled
         \end{itemize}
    \item “From <SI> for <Callsign> proposes Request on downstream frequency”
         \begin{itemize}
            \item Displayed when the next controller has sent a message requesting the track to be transferred to his frequency
            \item Left-clicking on the line manually removed
            \item Removed when a transfer is started
         \end{itemize} 
    \item “From <SI> for <Callsign> proposes Handover [HDG xxx] [DCT xxxxx] [SP xxx]”
        \begin{itemize}
            \item Displayed when there is a HOP in progress
            \item Left-clicking on the line opens the \textit{\titleref{menu:ctm}}
            \item Removed when the transfer is complete or cancelled
        \end{itemize} 
    \item “From <SI> for <Callsign> RTI [HDG xxx] [SP xxx] [ARC xx]”
        \begin{itemize}
            \item Displayed when an RTI message has been received
            \item Left-clicking on the line opens the \textit{\titleref{menu:ttm}}
        \end{itemize} 
    \item “From <SI> for <Callsign> TIP [HDG xxx] [SP xxx] [ARC xx]”
        \begin{itemize}
            \item Displayed when a TIP message has been received
            \item Left-clicking on the line opens the \textit{\titleref{menu:ttm}}
            \item Removed when the track becomes Assumed
        \end{itemize} 
    \item “From <SI> for <Callsign> proposes HOP/RTI/TIP Accepted”
        \begin{itemize}
            \item Displayed when an Accept message has been received to HOP, RTI or TIP
            \item Left-clicking on the line manually removes it
        \end{itemize} 
    \item “From <SI> for <Callsign> proposes RTI/TIP rejected [by timeout]”
        \begin{itemize}
            \item Displayed when a reject message has been received to RTI or TIP
            \item Left-clicking on the line manually removes it
        \end{itemize} 
    \item “From <SI> for <Callsign> proposes [COPN xxxxx] [PEL xxx]”
        \begin{itemize}
            \item Displayed when an entry coordination has been received
            \item Left-clicking on a value opens the EuroScope default menu to answer
            \item Removed when the track becomes Assumed
        \end{itemize} 
    \item “From <SI> for <Callsign> proposes [COPX xxxxx] [XFL xxx]”
        \begin{itemize}
            \item Displayed when an exit coordination has been received
            \item Left-clicking on a value opens the EuroScope default menu to answer an active coordination
        \end{itemize} 
    \item “From <SI>: <Callsign> will be squawking <ASSR>”
        \begin{itemize}
            \item Displayed when a previous controller assigns a new SSR code for a track
            \item Left-clicking on the line manually removes it
        \end{itemize}     
\end{itemize}


\subsection{Message Out Window}
\label{win:mow}

Global Menu -> ControlTools/Tools -> Message Out...

\includegraphics{img/mow.png}

The Message Out Window shows the coordination messages you have sent, sorted by time, with the newest ones at the top of the list. The messages will be automatically removed when the track becomes Unconcerned. To manually remove a line, left-click on it.

The available message types are:

\begin{itemize}
    \item“To <SI> for <Callsign> proposes goes to your frequency”
        \begin{itemize}
             \item Displayed when you transfer a track
        \end{itemize}
    \item“To <SI> Request <Callsign> on frequency”
        \begin{itemize}
             \item Displayed when you have sent a ROF message
        \end{itemize}
    \item“To <SI> for <Callsign> proposes Handover [HDG xxx] [DCT xxxxx] [SP xxx]”
        \begin{itemize}
             \item Displayed when you have sent a HOP
        \end{itemize}
    \item“To <SI> for <Callsign> RTI [HDG xxx] [SP xxx] [ARC xx]”
        \begin{itemize}
             \item Displayed when you have sent an RTI message
        \end{itemize}
    \item“To <SI> for <Callsign> TIP [HDG xxx] [SP xxx] [ARC xx]”
        \begin{itemize}
             \item Displayed when you have sent a TIP message
        \end{itemize}
    \item“To <SI> for <Callsign> proposes HOP/RTI/TIP Accepted”
        \begin{itemize}
             \item Displayed when you have sent an Accept message to HOP, RTI or TIP
        \end{itemize}
    \item“To <SI> for <Callsign> proposes RTI/TIP rejected [by timeout]”
        \begin{itemize}
             \item Displayed when you have sent a Reject message to RTI or TIP. “Rejected by timeout” will be sent automatically if the coordination is not answered within a specified time.
        \end{itemize}
    \item“To <SI> for <Callsign> proposes [COPN xxxxx] [PEL xxx]”
        \begin{itemize}
             \item Displayed when you have sent an entry coordination
        \end{itemize}
    \item“To <SI> for <Callsign> proposes [COPX xxxxx] [XFL xxx]”
        \begin{itemize}
             \item Displayed when you have sent an exit coordination
        \end{itemize}
    \item“To <SI> for <Callsign> proposes will squawk <ASSR>”
        \begin{itemize}
             \item Displayed when you have assigned a new SSR code and there is a next controller online for the track
        \end{itemize}
\end{itemize}
    
\subsection{Microphone Check Menu}
\label{menu:dlmcm}

Global Menu -> ControlTools/Tools -> CPDLC -> Microphone Check

\includegraphics{img/dlmicchk.png}

The Microphone Check menu is used to send a “CHECK STUCK MICROPHONE <frequency>” CPDLC message to all CPDLC connected aircraft when a stuck mic is suspected. The menu lists all frequencies selected for XMT TXT in EuroScope; left-clicking on one sends the messages. Manual frequency entry is not available.

The menu closes when a frequency is selected or the cursor leaves the menu area.

\subsection{CPDLC Current Message Window}
\label{win:dlcmw}

Global Menu -> ControlTools/Tools -> CPDLC -> Current Messages...

\includegraphics{img/dlcmw.png}

The CPDLC Current Message Window displays all sent and received CPDLC messages that have not been archived. Each line corresponds to one message, and contains the time the message was sent/received, the callsign of the aircraft, a filled triangle (pointing upward for uplink messages, downward for downlink messages) and the message text. If the message is too long to fit in the window, “…” is used to mark that there is more text in the message. Right-clicking on the message will open a small window that displays the entire message. The window will close automatically when the mouse cursor leaves its area.

The messages (for uplink clearances, also the responses) are color coded to display their status:

\begin{itemize}
    \item “Urgency” for CPDLC emergency messages that have not been replied to
    \item “CPDLC Failed” for failed uplink messages
    \item “CPDLC Unable” for uplink clearances replied to with “UNABLE”
    \item “CPDLC Standby” for uplink clearances replied to with “STANDBY”, and not timed out
    \item “CPDLC UM Clearance” for uplink clearances waiting for reply, and not timed out
    \item “CPDLC Pilot Late” for timed out uplink clearances
    \item “CPDLC Discarded” for discarded messages
    \item “CPDLC DM Request” for downlink requests waiting for controller reply, and not timed out
    \item “CPDLC Controller Late” for timed out downlink requests
    \item “Foreground” for other messages
\end{itemize}

Left-clicking on the current sorting option opens a popup to select the sorting order:

\begin{tabular}{l l}
    TIME/CS     & Messages sorted according to send/receive time (default option)\\
    DIALOG/CS   & Messages grouped to dialogues, dialogues sorted by time of first message\\
    CS/DIALOG   & Messages sorted by callsign, messages with same callsign sorted by time\\
\end{tabular}

Left-clicking on a message opens a popup to select some actions for the message:

\begin{tabular}{l l}
    Archive & Closes the message dialogue if open, archives all messages in that dialogue\\
    Discard & Closes the message dialogue and discards all the messages in it\\
\end{tabular}

If the message is “REQUEST LOGON”:

\begin{tabular}{l l}
    ACCEPT & Accepts the logon request\\
    UNABLE & Denies the logon request\\
\end{tabular}

For other messages:

\begin{tabular}{l l}
    Manual & Reply Opens a window (see below) to type a free text reply\\
\end{tabular}

Dialogues are never archived automatically, so make sure to manually archive dialogues when they are no longer relevant to keep the window from getting cluttered with messages.

Always reply to downlink requests using the relevant menu as it ensures correct formatting and type of message, and keeps the track label indications correct. The reply options here should only be used when a menu cannot be used, for example when the downlink was not parsed correctly and the request is therefore not shown on the track label.

\subsection{Manual Reply Window}
\label{win:dlmrw}

This window is used to send a manually composed reply to a CPDLC message (max 99 characters). Left-click on the area below the “Message text” label to enter the message and select one of the “Expected reply” options. Then left-click on “Send” to send the message.

The window closes automatically when the message is sent or the mouse cursor leaves the window area.

\begin{Warn}
    When replying to a downlink request using the Manual Reply Window, the track label is not updated accordingly. If the downlink had been recognized as a request, the request information is removed from the label, and if the reply is a clearance, it is not shown on the label, and the label values are not updated. Always use the label menus to answer a recognized downlink!
\end{Warn}

\subsection{CPDLC History Message Window}
\label{win:dlhmw}

Global Menu -> ControlTools/Tools -> CPDLC -> History Messages...

\includegraphics{img/dlhmw.png}

The CPDLC History Message Window contains CPDLC messages that have been archived from the CPDLC
Current Message Window.

\subsection{Cursor Lat/Long Window}
\label{win:latlon}

Global Menu -> ControlTools/Tools -> LAT/LONG...

\includegraphics{img/latlon.png}

Displays the latitude and longitude values of the cursor position.

\subsection{Weather Messages Window}
\label{win:wxmw}

Global Menu -> MET -> Messages...

\includegraphics{img/wxmsg.png}

The Weather Messages Window displays weather related messages. By default, METAR/SPECI messages are shown. To view other types of messages, left-click on the desired option button (the “LLF” and “AIREP” options are not available).

By default, with the “METAR/SPECI” option button chosen, this window displays the METARs you have requested from the VATSIM server (i.e. [F2] <icao>) and any METARs EuroScope requests automatically. Whenever a new METAR is received from the server it is added to the list (an old METAR is removed when a newer one is received from the same station). New METARs and TAFs are displayed in “Warning” color until the mouse cursor is positioned on them (for the decoded METAR, this applies only for the first row).

In addition, the window can display SIGMETs, SNOWTAMs and TAFs. The SIGMETs are retrieved when the “SIGMET” button is selected for the first time. Selecting the “SNOWTAM” or “TAF” option will open the \textit{\titleref{menu:ad}} where the desired stations must be selected. SNOWTAM data is downloaded together with the NOTAMs (as of writing this, they are not included in the data and hence unavailable for the plugin). If the plugin is configured to use a non-VATSIM METAR source, the METAR/SPECI list behaves the same way as the TAF list (stations must be selected from a list). The messages are automatically updated at specified intervals - the status bar at the bottom displays the remaining time to the next automatic update when the mouse cursor is over the window area.

The messages are sorted alphabetically by the station identifier in the list.

The messages can be viewed in three modes (decoded only available for METARs):

\begin{itemize}
    \item List (the default mode, showing one message per line)
    \item Single (showing only a single message)
    \item Single decoded (showing a single METAR in a decoded format)
\end{itemize}

To view a single message:

\begin{itemize}
    \item Left-click on a METAR/SIGMET/SNOWTAM/TAF in the list
    \item Left-click on a decoded METAR
\end{itemize}

To view a single METAR in the decoded format:

\begin{itemize}
    \item Right-click on a METAR in the list
    \item Right-click on a single METAR
\end{itemize}

To return to the list view:

\begin{itemize}
    \item Left-click on a single METAR/SIGMET/SNOWTAM/TAF
    \item Right-click on a decoded METAR
\end{itemize}

To remove a METAR/SIGMET/TAF from the window:

\begin{itemize}
    \item Display the METAR/SIGMET/TAF in the single or decoded mode
    \item Left-click on the “Del” button
    \item Left-click “Yes” in the confirmation window that opens (“No” cancels the operation)
\end{itemize}

\subsection{QNH/TL Window}
\label{win:qnhtl}

Global Menu -> MET -> QNH/TL...

\includegraphics{img/qnhtl.png}

The window displays the QNH values and corresponding transition levels for those airports that have a METAR displayed in the \textit{\titleref{win:wxmw}}. The transition level tables are defined in a data file, and if a table can’t be found for an airport in the list, a transition level will not be shown. The buttons in the window have no functionality.

\subsection{General Information Window}
\label{win:geninfo}

\includegraphics{img/geninfo.png}

The General Information Window displays basic information on the system state. The following functionality is implemented in the plugin:

\begin{tabular}{l l}
    mode                & “Free”, “Operational”, “Proxy”, “Replay” or “Training” depending on the connection method \\
    role                & Own controller ID (between the mode and the “Normal” item)\\
    QNH                 & Shows the local QNH value (below the “Normal” item when available)\\
    CPDLC Def           & Displays the state of the CPDLC Default setting\\
    FPASD               & Displays the state of the FPASD setting Alert\\
    Alert functions     & Status of the alert functions. The function name is shown in “Warning” color if selected off or there’s a fault in the data file.\\
    CCAMS               & Plugin selected as the code source:\\
                        & $\bullet$ “Urgency” color if the SSR data file contains no codes\\
                        & $\bullet$ “Warning” color if simulated traffic is not downloaded\\
                        & ESE file or fixed range selected as the code source:\\
                        & $\bullet$ “Warning” color\\
    AGCS                & “Warning” color if Hoppie datalink comms are failed\\
    MTCD                & “Warning” color if prediction time is set to zero    \\
\end{tabular}

\subsection{Document Viewer Window}
\label{win:dvw}

Global Menu -> Info -> Document Viewer...

\includegraphics{img/doc.png}

The Document Viewer Window is used to browse and search for information in specific data files

\begin{itemize}
    \item Aircraft - Information about aircraft types
    \begin{itemize}
        \item Type designator
        \item Wake turbulence category
        \item Description (Amphibian, Gyrocopter, Helicopter, Landplane, Seaplane, Tiltrotor)
        \item Engine count and type (Electric, Jet, Piston, Rocket, Turboprop/turboshaft)
        \item Manufacturer and model
    \end{itemize}
    \item Locations - Information about aerodromes
    \begin{itemize}
        \item Location indicator
        \item Location name
        \item (State/Territory)
    \end{itemize}
    \item Callsigns - Information about aircraft operators
    \begin{itemize}
        \item Three-letter designator
        \item Telephony designator
        \item (Aircraft operating agency and notifying state)
    \end{itemize}   
    \item Operators - Information about aircraft operators
    \begin{itemize}
        \item Three-letter designator
        \item Aircraft operating agency and notifying state
        \item (Telephony designator)
    \end{itemize}
\end{itemize}

The information can be filtered using the “Filter” box. The list will only display lines containing the entered text string (case insensitive).

\subsection{NOTAM List Window}
\label{win:notamlw}

Global Menu -> Info -> NOTAM...

\includegraphics{img/notamlist.png}

The NOTAM List Window displays a list of received NOTAMs. The NOTAMs are retrieved when the window is first opened and the list will take a couple of seconds to populate. The NOTAMs are automatically updated every two hours. Each NOTAM shows the following information:

\begin{itemize}
    \item Location ID (ICAO designator)
    \item Serial number
    \item Type (“N” = new, “R” = replaces earlier NOTAM, “C” = cancels earlier NOTAM)
    \item Abbreviated form of the NOTAM message contents
\end{itemize}

To see the actual NOTAM contents, left-click on a NOTAM line. To return back to the NOTAM list, left-click on the single NOTAM.

By default, the window displays all received NOTAMs. Left-clicking on the “All” label opens a menu where the displayed NOTAMs can be filtered by their validity periods. The other available options are “Today” and “Tomorrow”.

The NOTAM List is automatically displayed at startup in order to fetch the current FUA. It may be closed after loading.

\subsection{Aerodrome Window}
\label{win:adw}

Global Menu -> Info -> Aerodrome...

\includegraphics{img/adinfo.png}

The Aerodrome Window displays information about the selected airport.

Left-clicking on the airport ICAO code opens the \textit{\titleref{menu:ad}} to select another airport, and left-clicking on the “NOTAM” button opens the \textit{\titleref{win:notamlw}}, showing only the NOTAMs for the selected airport.

Below the header, the latest METAR and TAF for the airport are displayed. The QNH and TRL in the header are based on the METAR data.

The bottom part of the window displays the arrival and departure runway allocation at the selected airport. The runway selections are read-only in this window.

\subsection{LFUNC Frequency Plan Window}
\label{win:lfunc}

Global Menu -> Info -> LFUNC Frequency Plan...

\includegraphics{img/lfunc.png}

The LFUNC Frequency Plan Window displays the currently online controllers and their primary frequencies as well as their CPDLC logon callsigns where applicable.

\subsection{Notepad Window}
\label{win:note}

Global Menu -> Messages/MSG -> Notepad...

\includegraphics{img/note.png}

The Notepad Windows can be used to display any user entered text. Multiple Notepad Windows can be opened simultaneously. To enter new text or edit the existing one, click on the window area. The text will be automatically wrapped, and if necessary, the “|” character can be used to force a line break. If the window is not large enough to fit all the entered text, it will display “…” in the end to indicate that there is more information.

\subsection{Personal Queue Window}
\label{win:pqw}

Global Menu -> Messages/MSG -> Personal Queue...

\includegraphics{img/pq.png}

The Personal Queue Window displays warning messages related to the plugin’s operation: high priority messages informing about potential critical failures in the plugin code, and low priority messages informing about faults in the plugin’s external data files or timeout alerts for coordination messages.

The window currently only displays “ALERT” type messages, and the origin for them is always empty. The time field displays the UTC time when the currently viewed message was created.

The high priority messages are always displayed first. Only when there are no more high priority messages in the list, are the low priority ones shown. To acknowledge a message, click on the “ACK” button. This removes the message from the list and displays the next one. The “Next” button moves the currently viewed message to the back of the list and displays the next message of the same priority.

\subsection{ATC / Primary Frequency Messages Window}
\label{win:atcmw}
\label{win:pfmw}

Global Menu -> Messages/MSG -> ATC Messages...
Global Menu -> Messages/MSG -> Prim Freq Messages...

\includegraphics{img/atc.png}

These windows display up to 99 last messages transmitted on the relevant text channel (“ATC” or the primary frequency). Each message line displays a time stamp, the sender callsign (blank if you) and the message itself.

New incoming messages are displayed in “Warning” color until left-clicked to mark them as read. Left-double-clicking on any message will mark all messages in the window read. The windows do not resize automatically to show all the messages in them, but the number in the title bar shows the total number of messages in the window. If the window is not wide enough to fit a complete message, it will display “…” in the end to indicate that there is more information. Holding a mouse button down on a message will display the entire message.

\subsection{NAT Track Messages Window}
\label{win:nattmw}

Global Menu -> Messages/MSG -> NAT Track Messages...

\includegraphics{img/nat.png}

The NAT Track Messages Window displays the downloaded track messages. The messages are downloaded when the window is opened the first time and then updated hourly. The tracks extracted from the messages can be displayed on the radar screen – they are placed to a “NAT” folder in the \textit{\titleref{win:mapsw}}. Any named waypoints in the tracks that cannot be found in the active sector file are just skipped so the displayed tracks may not be accurate. The track letter is added to the name of the first and last waypoints in parentheses, so a missing track letter is a sure sign of at least some waypoints missing.

\subsection{Safety Nets Status Window}
\label{win:snsw}

Global Menu -> Status/STS -> Safety Nets Status...

\includegraphics{img/snets.png}

Allows setting the status for the STCA, MSAW, APW and AIW systems. The “On/Off” buttons control the corresponding system’s status.

Below them, there are four entry boxes for callsigns to exclude specific callsigns from all the safety nets. The “*” wildcard can be used to match multiple callsigns. It causes all the callsigns that match up to the “*” to be a match (i.e. “ABC*” will match all callsigns that start with “ABC”, but “*ABC” will match all callsigns as any characters after the “*” will be ignored).
Below the callsign fields, there are four SSR code boxes that can be used to exclude specific SSR codes from all the safety nets. The entered values must be 1-4 octal digits, and the system will match the number of digits entered (i.e. “2000” will match only code 2000, whereas “20” will match all codes in the range 2000-2077).
If there are STCA inhibition areas defined in the area data file, they will be listed in the area in the bottom part of the window. The area vertical limits (displayed in 100’s of feet or meters+“m” depending on system units) can be edited by clicking on the values, and the area activation can be toggled by clicking on the area status.
All changes to the window must be applied using the buttons in the bottom of the window to take effect.

\begin{tabular}{l l}
    Ok      & Applies the changes if any, closes the window\\
    Apply   & Applies the changes\\
    Cancel  & Cancels the changes\\
\end{tabular}

\subsection{Divergence Detection Status Window}
\label{win:ddsw}

Global Menu -> Status/STS -> Divergence Detection Status...

\includegraphics{img/divsts.png}

Allows setting the status for the RAM and CLAM alerting. The “On/Off” buttons control the corresponding system’s status. All changes must be applied using the buttons below to take effect.

\begin{tabular}{l l}
    Ok      & Applies the changes if any, closes the window\\
    Apply   & Applies the changes\\
    Cancel  & Cancels the changes\\
\end{tabular}

\subsection{MTCD Status Window}
\label{win:mtcdsw}

Global Menu -> Status/STS -> MTCD Status...

\includegraphics{img/mtcd.png}

Allows setting the status for the MTCD system. The “On/Off” button controls the system’s status.

If there are MTCD inhibition areas defined in the area data file, they will be listed in the area below the “On/Off” button. The area vertical limits (displayed in 100’s of feet or meters+”m” depending on system units) can be edited by clicking on the values, and the area activation can be toggled by clicking on the area status.

The rest of the inhibition settings affect both MTCD and SAP systems:

Below the inhibit areas, there are four SSR code boxes that can be used to exclude specific SSR codes from MTCD/SAP processing. The entered values must be 1-2 octal digits, and the system will match the number of digits entered (i.e. “2” will match codes 2000-2777, whereas “20” will match codes 2000-2077).

Below the SSR codes, there are four ATYP boxes to exclude specific aircraft types. The entered text strings must be exact ICAO aircraft type designators (no partial matches or wildcards).

The flight type inhibit is not available due to network restrictions.

VFR or IFR flights can be excluded by selecting the respective “Inhibited Flight Rules” button to “On”.

Finally, there are four entry boxes to exclude specific callsigns. The “*” wildcard can be used to match multiple callsigns. It causes all the callsigns that match up to the “*” to be a match (i.e. “ABC*” will match all callsigns that start with “ABC”, but “*ABC” will match all callsigns as any characters after the “*” will be ignored).

All changes must be applied using the buttons below to take effect.

\begin{tabular}{l l}
    Ok      & Applies the changes if any, closes the window\\
    Apply   & Applies the changes\\
    Cancel  & Cancels the changes\\
\end{tabular}

\subsection{Runway in Use Window}
\label{win:riuw}

Global Menu -> Status/STS -> Runway In Use...
Radar screen -> middle-click on Airport symbol drawn by the plugin’s “AD\_Hotspots” map...

\includegraphics{img/rwy.png}

The Runway In Use Window displays the arrival and departure runway allocation at the selected airport. Left-clicking on the airport ICAO code opens the \textit{\titleref{menu:ad}} to select another airport. The runway selections are read-only in this window.

\subsection{Operations Rate Window}
\label{win:orw}

\includegraphics{img/opsrate.png}

The Operations Rate Window displays the predicted hourly operations rate at a specified airport. The data is displayed in 5-minute steps and shown up to 5 hours into the future. The arrival prediction is based on aircraft tracks as calculated by EuroScope. The departures are predicted to depart immediately if their ground status is set to “DEPA”, earliest in 5 minutes if “TAXI” and earliest in 10 minutes if “PUSH”, with a maximum departure rate of one aircraft per minute.

A number of these windows can be opened to simultaneously view multiple combinations of airport, departure/arrival state and runway(s).

In the area below the title bar, on the left is the ICAO identifier of the airport whose traffic is being monitored (a gray label “ICAO” is shown if no airport is selected yet).

The “Dep” and “Arr” buttons control whether departures and/or arrivals should be shown on the display (button background is shown in “Selected” color if selected on).

The “RWY” label allows entering one or more runway identifiers to filter traffic based on the assigned runway.

The numbers on the right side are the caution and warning limits. The rates are color coded so that a rate at or below the caution limit will be shown in “Information” color, a rate above that but at or below the warning limit in “Warning” and a rate above that in “Urgency”. When both arrivals and departures are selected for display, the departures are shown with a hatched color.

\begin{tabular}{l l}
    ICAO        & Enter airport identifier\\
    Dep         & Toggle departures on/off\\
    Arr         & Toggle arrivals on/off\\
    RWY         & Enter runways\\
    XX / XX     & Enter caution and warning limits\\
    Sliders     & Change the rate number and time scales  \\  
\end{tabular}

\subsection{Predicted Traffic Window}
\label{win:ptw}

Global Menu -> Status/STS -> Supervisory -> Predicted Traffic...

The Predicted Traffic Window shows the number of aircraft that are predicted to be inside a specified controller’s airspace. The data is displayed in 5-minute steps and shown up to 5 hours into the future. The prediction is based on the sector ownership and the aircraft tracks are as calculated by EuroScope.

A number of these windows can be opened to simultaneously view multiple controllers’ situation. In the area below the title bar, the left side shows the controller ID whose traffic is being monitored (a gray label “ID” is shown if no controller ID is selected yet). Left-click on the text to enter a new ID.
T
he numbers on the right side are the caution and warning limits. To change them, left-click on them and re-enter in the same format (warning can’t be lower than caution; numbers must be separated by a forward slash). The traffic numbers are color coded so that a number at or below the caution limit will be shown in “Information” color, a number above that but at or below the warning limit in “Warning” and a number above that in “Urgency”.

The two sliders change the traffic number and time scales.

\subsection{Runway Approach Line Window}
\label{win:rapw}

Global Menu -> Status/STS -> RWY line display... (opens the \textit{\titleref{menu:ad}} for airport selection)
Radar screen -> right-click on Airport symbol drawn by the plugin’s “AD\_Hotspots” map...

\includegraphics{img/rwydsp.png}

To open the window from the radar screen, right-click on an airport symbol (+) on the radar screen (for the click spots to be active, the “AD\_Hotspots” map in the “Aerodromes” folder of the \textit{\titleref{win:mapsw}} must be active). Only those airports with runways specified have active click spots.

The window will open below the mouse cursor, or will be repositioned there if already open.

The window contains selection buttons to toggle the display of the approach lines for the airport’s runways, and a listing of the currently active runways for the airport. The “ARR RWL DISPLAY” option toggles the automatic display of all approach lines for runways that are selected active for arrival. The setting is global for all airports. The runway approach line is by default 20nm long and has 5 distance markers at 2nm intervals. The color of the line is different depending on whether the runway is active for arrivals or not.

\subsection{Tactical Info Window}
\label{win:tiw}

“Open Tactical Info Window” tag function

\includegraphics{img/tiw.png}

The Tactical Info Window displays coordinated tactical data (AHDG, ASP and/or ARC). Proposed data is in “Proposition”, rejected data in “Warning” and accepted data in sector state color. This window only displays the information; to answer or apply the data use the \textit{\titleref{menu:ttm}}.

\subsection{Pre-Departure Clearance Window}
\label{win:dlpdcw}

“Open PDC Window” tag function

\includegraphics{img/dcl.png}

The Pre-Departure Clearance Window is used to issue departure clearances, either via R/T or datalink. The following data fields are displayed for review:

\begin{tabular}{l l}
    RWY     & Departure runway (left-click to open RWY setup popup list)\\
    SID     & Assigned SID (left-click to open SID setup popup list)\\
    AHDG    & Assigned heading (left-click to open \textit{\titleref{menu:ahdg}})\\
    NPT     & Next route point (left-click to edit)\\
    CFL     & Cleared level (left-click to open \textit{\titleref{menu:cfl}})\\
    ASSR    & Assigned transponder code (left-click to open \textit{\titleref{menu:assr}})\\
    START   & Start-up clearance (left-click to toggle between “YES” and “NO”)\\
    NFREQ   & Next frequency (left-click to edit)\\
    DFREQ   & Departure frequency (left-click to edit)\\
    RMK     & Free text remarks to send with a datalink clearance (left-click to edit)\\
\end{tabular}

The NPT, NFREQ, DFREQ, START and RMK fields are only displayed if a datalink clearance has been requested and the clearance format (defined in the CPDLC data file) includes them.

The RWY field text is displayed in “Proposition” color if the clearance has not yet been issued and the selected runway is not active for departure. The CFL level background is displayed in “Warning” color for DEL/GND/TWR/APP/DEP controllers when it is above XFL, and for CTR/FSS controllers when it is above PEL.

When the datalink clearance format includes the possibility to switch between “track” and “heading” modes, the “AHDG” label is displayed as a button. Left-click on it to toggle between “AHDG” and “TRACK”.

When the RMK field contains text, the window displays the full text if it is not longer than 8 characters, otherwise the 7 first characters + “…”. Left-click to edit the text.

When the aircraft has requested a clearance via datalink, the two buttons in the window are:

\begin{tabular}{l l}
    Send DCL    & Sends a departure clearance via datalink, closes the window\\
    Voice       & Sends an error message “REVERT TO VOICE PROCEDURE”, closes the window\\
\end{tabular}

When the aircraft has not requested a clearance via datalink, not enough data is entered, or the controller has used the “Voice” option above to abort the datalink clearance process, the buttons are:

\begin{tabular}{l l}
    Ok      & Sets the clearance flag “On”, closes the window\\
    Cancel  & Sets the clearance flag “Off”, closes the window\\
\end{tabular}

The “Send DCL” and “Ok” buttons are active when at least the “RWY”, “CFL”, “ASSR”, all displayed frequency fields and the “NPT” field, if displayed, contain data.

\subsection{Departure Coordination Window}
\label{win:dcw}

“Toggle EST/DEP/ABT” function (when clearance flag not set and ground state not “DEPA”)

\includegraphics{img/dcw.png}

The Departure Coordination Window can be used to set the clearance flag. The window displays the aircraft callsign, departure runway and SID and allows setting the CFL, ASSR and EOBT values.

\begin{tabular}{l l}
    CFL     & Opens the CFL menu\\
    ASSR    & Opens the ASSR menu\\
    EOBT    & Opens the Time menu\\
    Cancel  & Disregards any changes, closes the window\\
    Ok      & Applies any changes, sets the clearance flag and closes the window\\
\end{tabular}






\section{Lists}

\section{Safety Nets}

\section{Monitoring Aids}

\section{Flight Plan Conflict Probe}

\appendix

\chapter{Label field descriptions}

\chapter{Color Values}

\chapter{Keyboard Shortcuts}

\bibliographystyle{unsrt}
\bibliography{refs}

\end{document}

